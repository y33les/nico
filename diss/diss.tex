% The master copy of this demo dissertation is held on my filespace
% on the cl file serve (/homes/mr/teaching/demodissert/)

% Last updated by PMY on 3 March 2012

\documentclass[12pt,twoside,notitlepage,xetex]{report}

\usepackage{a4}
\usepackage{verbatim}
\usepackage{epsf}
\usepackage{sectsty}

\usepackage{url}
\usepackage{parskip}
\usepackage{pdfpages}
\usepackage{wrapfig}
\usepackage{subfig}
\usepackage[normalem]{ulem}
\usepackage{caption}
\usepackage{color}

\usepackage{graphicx}
\usepackage{fontspec,xunicode}
\defaultfontfeatures{Mapping=tex-text, Numbers=Monospaced}% ,Scale=MatchLowercase}% , Numbers=OldStyle}
\setmainfont[Scale=1]{Sabon MT Std}% {Linux Libertine O}
\setsansfont{Myriad Pro Light}% {Roboto Bold}
\setmonofont[Scale=0.8]{Monaco}% {Droid Sans Mono}
\allsectionsfont{\sffamily}

\newfontinstance\bigsf[Color=000000,Scale=1.25]{Myriad Pro Light}

\definecolor{red}{rgb}{0.80,0.00,0.00}
\definecolor{green}{rgb}{0.00,0.80,0.00}
\definecolor{blue}{rgb}{0.00,0.00,0.80}

\input{epsf}                            % to allow postscript inclusions
% On thor and CUS read top of file:
%     /opt/TeX/lib/texmf/tex/dvips/epsf.sty
% On CL machines read:
%     /usr/lib/tex/macros/dvips/epsf.tex

\raggedbottom                           % try to avoid widows and orphans
\raggedright                            % left-aligned, not justified
\sloppy
\clubpenalty1000%
\widowpenalty1000%

\addtolength{\oddsidemargin}{6mm}       % adjust margins
\addtolength{\evensidemargin}{-8mm}

\renewcommand{\baselinestretch}{1.1}    % adjust line spacing to make
                                        % more readable

\renewcommand{\captionfont}{\footnotesize}

\begin{document}

\bibliographystyle{unsrt}% {plain}


%%%%%%%%%%%%%%%%%%%%%%%%%%%%%%%%%%%%%%%%%%%%%%%%%%%%%%%%%%%%%%%%%%%%%%%%
% Title


\pagestyle{empty}

\hfill{\LARGE \bf \sffamily Philip Yeeles}

\vspace*{60mm}
\begin{center}
\Huge
{\bf \bigsf Nico: An Environment for Mathematical Expression in Schools} \\
\vspace*{5mm}
{\sffamily Computer Science Tripos} \\
\vspace*{5mm}
{\sffamily Selwyn College} \\
\vspace*{5mm}
{\sffamily \today}  % today's date
\end{center}

\cleardoublepage

%%%%%%%%%%%%%%%%%%%%%%%%%%%%%%%%%%%%%%%%%%%%%%%%%%%%%%%%%%%%%%%%%%%%%%%%%%%%%%
% Proforma, table of contents and list of figures

\setcounter{page}{1}
\pagenumbering{roman}
\pagestyle{plain}

\chapter*{Proforma}

{\large
\begin{tabular}{ll}
\bf Name:               & Philip Yeeles                                                   \\
\bf College:            & Selwyn College                                                  \\
\bf Project Title:      & Nico: An Environment for Mathematical\\
                        & Expression in Schools \\
\bf Examination:        & Computer Science Tripos, May 2012                               \\
\bf Word Count:         & TBC\footnotemark[1]
(well less than the 12000 limit) \\
\bf Project Originator: & P.~M.~Yeeles (\verb¬pmy22¬)                                    \\
\bf Supervisors:        & Dr S.~J.~Aaron (\verb¬sja55¬), A.~G.~Stead (\verb¬ags46¬)     \\
\end{tabular}
}
\footnotetext[1]{This word count was computed
by {\tt detex diss.tex | tr -cd '0-9A-Za-z $\tt\backslash$n' | wc -w}
}
\stepcounter{footnote}


\section*{Original Aims of the Project}

The aim of the project was to develop an application in the Clojure programming
language which would allow users to express mathematical calculations using a
graphical notation.  The software was to be able to generate an abstract syntax
tree from the graphical notation, evaluate it and pass the results back to the
application in under 300ms.  An extension to the project was to conduct a user
study to evaluate the utility of the software.
%
% To write a demonstration dissertation\footnote{A normal footnote without the
% complication of being in a table.} using \LaTeX\ to save
% student's time when writing their own dissertations. The dissertation
% should illustrate how to use the more common \LaTeX\ constructs. It
% should include pictures and diagrams to show how these can be
% incorporated into the dissertation.  It should contain the entire
% \LaTeX\ source of the dissertation and the Makefile.  It should
% explain how to construct an MSDOS disk of the dissertation in
% Postscript format that can be used by the book shop for printing, and,
% finally, it should have the prescribed layout and format of a diploma
% dissertation.


\section*{Work Completed}

I have successfully designed and implemented the application detailed in the
previous section.  That is, I have developed an application in which it is
possible to express calculations using a graphical notation, that generates an
abstract syntax tree from the language and that is able to parse the tree and
return the results in under 300ms.  I have also conducted a user study to assess
whether or not the software is actually of use with regard to mathematics
education.
%
% All that has been completed appears in this dissertation.

\section*{Special Difficulties}

Learning the Clojure programming language.
%
% Learning how to incorporate encapulated postscript into a \LaTeX\
% document on both CUS and Thor.

\newpage
\section*{Declaration of Originality}

I, Philip Michael Yeeles of Selwyn College, being a candidate for Part
II of the Computer Science Tripos, hereby declare that this dissertation
and the work described in it are my own work, unaided except as may be
specified below, and that the dissertation does not contain material
that has already been used to any substantial extent for a comparable
purpose.

\bigskip
\leftline{Signed}

\medskip
\leftline{Date \today}

\cleardoublepage

\tableofcontents

\listoffigures

\newpage
\section*{Acknowledgements}
My thanks to Luke Church for his advice regarding user studies, and to Alistair
Stead and Sam Aaron for their encouragement and patience.
%
% This document owes much to an earlier version written by Simon Moore
% \cite{Moore95}.  His help, encouragement and advice was greatly
% appreciated.

%%%%%%%%%%%%%%%%%%%%%%%%%%%%%%%%%%%%%%%%%%%%%%%%%%%%%%%%%%%%%%%%%%%%%%%
% now for the chapters

\cleardoublepage        % just to make sure before the page numbering
                        % is changed

\setcounter{page}{1}
\pagenumbering{arabic}
\pagestyle{headings}

\chapter{Introduction}

The aim of this project has been to design and develop a notation and
accompanying application to act as a learning aid for pre-algebra arithmetic by
increasing visibility, reducing the number of hidden dependencies and making the
flow of data obvious to the user.  I have successfully developed such a system,
intended initially for pupils in Year 5 (though extensible, through the creation
of alternative question sets, to other age groups), and, as an extension,
conducted a user study to assess its utility.

In this chapter, I will discuss my motivations for choosing this project, the
pros and cons of the handwritten system it is attempting to augment, the
technical challenges involved in developing such a system and related work that
has previously been conducted with similar goals.

\section{Motivations}
% evaluation of problems with writing it down on paper
% cogdims of handwritten method(s)
% system requirements (cogdims)

My motivations behind this project lay in the limitations of the handwritten
approach to solving mathematical problems that I had observed both in my own
learning and in my own teaching experience.  What follows is an evaluation of
the pros and cons of the handwritten method of performing arithmetic
calculations according to Blackwell and Green's ``Cognitive Dimensions''
framework \cite{Blackwell1998}, and a discussion of the properties a useful
alternative notation should have.

\pagebreak

\subsection{Evaluation of the Handwritten Approach}

\begin{wrapfigure}{r}{0.5\textwidth}
\begin{center}
\subfloat[]{
\parbox{2cm}{
\begin{center}
{\small
24+35=59\\
12+48=60\\
59+72=131\\
1+60=61\\
131+61=192}
\end{center}}}
\subfloat[]{
\parbox{2cm}{
\begin{center}
{\small
24+35={\color{green}59}\\
12+48={\color{blue}60}\\
{\color{green}59}+72={\color{green}131}\\
1+{\color{blue}60}={\color{blue}61}\\
{\color{green}131}+{\color{blue}61}=192}
\end{center}}}
\subfloat[]{
\parbox{4cm}{
\begin{center}
{\small
24+35= {\color{red}\sout{59}} {\color{green}49}\\
12+48={\color{blue}60}\\
{\color{red}\sout{59}} {\color{green}49}+72= {\color{red}\sout{131}} {\color{green}121}\\
1+{\color{blue}60}={\color{blue}61}\\
{\color{red}\sout{131}} {\color{green}121}+{\color{blue}61}= {\color{red}\sout{192}} 182}
\end{center}}}
\end{center}
\caption{Illustrating the hidden dependencies and viscosity inherent in pre-algebra, handwritten arithmetic.  The original calculation is shown in (a), has its otherwise-hidden dependencies highlighted in (b), and is altered slightly in (c).}
\end{wrapfigure}
% cogdims
% taught method without being taught *why* it works
% high viscosity - have to cross things out or start again
% high repetition viscosity
% low knock-on viscosity
% viscosity is acceptable in transcription and incrementation, but harmful to modification and exploration
% hidden dependencies -
% premature commitment - made worse by high repetition viscosity; have to start again if a mistake is made
% premature commitment is certainly very high for traditional calculators
% algebra is abstraction-hungry, arthimetic is abstraction-hating, though can use secondary notation to construct abstractions
% no abstraction barrier to arithmetic though
% secondary notation - notes around calculations
% visibility - visible, but only juxtaposable if juxtaposed at the start (premature commitment) or with the help of secondary notation
There are a number of problems with handwritten, pre-algebra arithmetic that
this project seeks to rectify.  First of all, the fact that it is handwritten
entails a high level of viscosity: it is difficult to make changes to a written
calculation without sacrificing clarity.  In particular, there is a lot of
repetition viscosity involved in the modification of an existing piece of work;
if a number is changed that is used in several calculations, then it is time-
consuming to change it everywhere it appears in the working.  If several
calculations are dependent upon each other, then this entails a lot of knock-on
viscosity in recalculating each stage after changing the number.  This is
exacerbated by the hidden dependencies between chained calculations in
handwritten arithmetic (\emph{Fig. 1.1}).

\subsection{Requirements of a Replacement System}
% cogdims!!
% need to develop useful notation - early studies, talk about other notations we came up with, notations looked at with alistair (get link)
% hidden dependencies - nico shows a data flow representation (p19, blackwell1998) to make dependencies explicit

% TODO: this bit.  not even sure if it should be in the intro...

To improve upon the standard approach of listing the steps comprising a
calculation, a system must acknowledge and try to overcome the drawbacks listed
above.  To this end, I have designed and developed a system that aims to
eliminate in particular the many hidden dependencies of traditional, handwritten
arithmetic.

\section{Technical Challenges}
% need first-class functions
% need to be able to pass expressions around to evaluate at will: hence functional, hence lisp - homoiconicity
% also need good gui libraries for app - familiar with java/swing so that works (esp. with seesaw), also the option of jfx2 and swt/guiftw
% difficulty of learning clojure over summer in preparation

Developing such an application comprises two main challenges: developing a
backend that is capable of creating, storing, editing, deleting, evaluating,
nesting and calculations, and a graphical, user-facing frontend that is able to
render calculations into the devised notation, and allow the user to perform
operations upon the notation that affect the underlying calculation.

I chose to use the Clojure language as it provided many features that would
prove to be useful over the course of the application's development.  As a
dialect of LISP, Clojure is a homoiconic programming language -- that is, a
programming language in which code is represented as a data structure -- which
made passing around and performing operations upon calculations themselves,
rather than just their results, considerably easier.  A calculation can simply
be represented as a piece of code, which can then be utilised as needed.

As the user experience is so crucial to the success of the application, it was
also important that there be well-established GUI libraries available.  Clojure
runs on the Java Virtual Machine (JVM), which puts Java's considerable standard
library at one's disposal, whilst still being able to program in a LISP.  As I
am familiar with Java and the Swing GUI libraries, it was advantageous to be
able to leverage this knowledge in designing the application's interface.

\section{Previous Work}
% scrubbing calculator http://worrydream.com/ScrubbingCalculator/
% soulver http://www.acqualia.com/soulver/
% neither really targeted at education
% cool stuff at http://worrydream.com/KillMath/ though
% maybe refer to this? http://betterexplained.com/articles/rethinking-arithmetic-a-visual-guide/
% also, this: http://www.ralph-abraham.org/articles/Blurbs/blurb126.shtml
% also, check out espresso and education city stuff, log in to fronter via st. mary's site, uname jyeele1.314, pword b3ach7
% plenty of work been done in making computer maths like paper maths, but not so much the reverse, case in point: http://www.macresearch.org/showcase-review-pi-cubed-iphone-ipod-touch
% also a lot of crap maths games that are essentially a series of qs in disguise, see http://www.time4learning.com/curriculum/try_demos.html
% nico is unabashedly *not* a game, rather a tool, much like a pen or calculator
% problem with calculators?

There already exists a wide variety of educational software for mathematics, but
much of this is in the form of ``games'', in which a series of mathematical
problems to be solved is poorly disguised as a game -- indeed, such problems
would be more accurately said to be embedded into a game, rather than becoming
the game themselves.  Thus, the object becomes not to solve the problems, but to
play the game that happens to surround the problems.  Such software also does
not often offer any means of solving the problems, other than the traditional
pen-and-paper method (with a piece of paper next to the computer screen), or
the mental approach.  Hence, what the user is then presented with is essentially
a game and a worksheet, awkwardly interleaved.  In some cases, it is even
possible for the user to simply press arbitrary buttons until they pass the
questions, effectively removing the maths element of the game and replacing it
with a series of short breaks in gameplay. % citations!

There also exist a few applications intended to represent calculations on a
computer in novel ways.  A relatively common approach to this has been to try
to make on-screen calculations more like on-paper calculations.  \emph{Pi Cubed}
takes this approach by trying to make complex calculations appear as they would
be written in an exam or exercise book \cite{PiCubed}.  \emph{Soulver},
conversely, tries to achieve this by simulating ``back-of-the-envelope''
calculations, whereby notes in English augment the calculation \cite{Soulver}.
Another approach is that of the \emph{Scrubbing Calculator} \cite{ScrubCalc},
which extends the \emph{Soulver}-style environment by helping the user to solve
equations by dragging values to increase and decrease them, showing how changing
a value affects the overall result.  Values can be linked by dragging a line
between them, which means that they are two instances of the same value -- hence
dragging one changes the value at every location in which it appears.  This is a
neat means of visualising equations, but it, too, is not intended for use in
education, and still requires the user to be able to formulate some kind of
equation.  The \emph{Scrubbing Calculator} is more a tool for facilitating
algebraic understanding, as opposed to arithmetic understanding; indeed, it is
inherently a {\bf calculator}, and so does not encourage thinking about how to
work out the arithmetic parts of a calculation manually.

\section{Summary}
% need a system that incorporates these features: foo, bar, baz, quux

Existing educational ``games'' for mathematics either have too much focus on
being a game, rather than helping to learn mathematics, or are such that the
mathematical element is circumventable.  There exists software to aid in
calculation and arithmetic by representing it clearly, but it is not intended
for educational use, and often its purpose is to make on-screen calculations
appear as one would handwrite them.

There is a niche for a tool for use in education that represents calculations in
a visual manner, with a particular focus on making the method by which arithmetic
problems are solved clear.  My project aims to provide an environment in which
the user can explore the many ways in which a problem can be solved using a
novel graphical notation.

\cleardoublepage



\chapter{Preparation}

In this chapter I will discuss the work that was done prior to beginning the
project proper.  This includes learning the Clojure programming language and
researching and auditioning graphical metaphors for calculation.  This chapter
comprises a requirements analysis, followed by an overview of the system
architecture and a discussion of the additional tools used in the development of
the project.

\section{Requirements Analysis}
% http://www.cs.fsu.edu/~lacher/courses/COP3331/rad.html
% http://www.nd.gov/itd/files/services/pm/requirements-analysis-guidebook.pdf

\subsection{Current System}
% move evaluation of handwritten arithmetic from chapter 1 here?

\subsection{Proposed System}

\subsubsection{Overview}

\subsubsection{Functional Requirements}

To be an improvement upon the current system detailed above, the new system must
satisfy the following properties:--
\begin{itemize}
\item lol
\end{itemize}

\subsubsection{Non-Functional Requirements}

The system must also satisfy a number of requirements outside of its basic
functionality.  These are listed below.
\begin{itemize}
\item lol
\end{itemize}

\subsubsection{Use Case}
% diagram?  see wikipedia "use case"

% \section{System Overview}
% % do we need if in req. analysis?
%
% lol

\section{User Interface}
% section rather than subsection?  def. section if we don't have system overview

lol

\section{Additional Tools}

lol

\subsection{Third-Party Tools}

What follows is a list of the third-party tools that were used in the
development of the project.
\begin{itemize}
\item Ubuntu Linux 10.04, Arch Linux 2010.05, Microsoft Windows 7
\item Clojure 1.2.0
\item Leiningen 1.6.1.1
\item OpenJDK 6
\item Seesaw 1.3.1-SNAPSHOT
\item swank-clojure 1.3.4-SNAPSHOT
\item GNU Emacs 23.1.1
\item A modified version of Overtone's Emacs configuration \cite{OvertoneEmacsD}, including:--
\begin{itemize}
\item SLIME/SWANK (revision as of 15/10/2009)
\item clojure-mode 1.11.5
\item undo-tree 0.3.3
\end{itemize}
\item Git 1.7.0.4
\item GitHub
\end{itemize}

\section{Summary}


\cleardoublepage
\chapter{Implementation}

lol

\section{Backend}
% calculations as data structures (lists to be evaled)
% rendering engine
% questions and marking
% question syntax
% questions as data structures (lists to be evaled)
% the fact that they are essentially answers means that answers can easily be compared to them, as well as manipulated and stuff - case in point, all the question-highlighting bullshit
% interactivity (300ms rule, see proposal for citation)

\section{User Interface}
% target audience - children (year 5), but also applications in remedial adult learning (i.e. can't be too childish)
% hence must be appealing, clear, easy-to-read, not too verbose

\subsection{Metaphor}
% graphical language itself

\subsection{Application}
% application in which the language is manipulated

\section{Summary}




\cleardoublepage
\chapter{Evaluation}

lol

\section{Summary}


\cleardoublepage
\chapter{Conclusions}

lol




\cleardoublepage

%%%%%%%%%%%%%%%%%%%%%%%%%%%%%%%%%%%%%%%%%%%%%%%%%%%%%%%%%%%%%%%%%%%%%
% the bibliography

\addcontentsline{toc}{chapter}{Bibliography}
\bibliography{refs}
\cleardoublepage

%%%%%%%%%%%%%%%%%%%%%%%%%%%%%%%%%%%%%%%%%%%%%%%%%%%%%%%%%%%%%%%%%%%%%
% the appendices
\appendix

\chapter{Project Proposal}

The original project proposal follows.

\includepdf[pages=-]{proposal_rev04.pdf}
% \input{propbody}

\end{document}
