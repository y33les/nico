% The master copy of this demo dissertation is held on my filespace
% on the cl file serve (/homes/mr/teaching/demodissert/)

% Last updated by PMY on 3 March 2012

\documentclass[12pt,twoside,notitlepage,xetex]{report}

\usepackage{a4}
\usepackage{verbatim}
\usepackage{epsf}
\usepackage{sectsty}

\usepackage{url}
\usepackage{parskip}
\usepackage{pdfpages}
% \usepackage{wrapfig}
\usepackage{subfig}
\usepackage[normalem]{ulem}
\usepackage{caption}
\usepackage{color}
\usepackage{float}
% \usepackage{lettrine} % dropped capitals at start of chapter
\usepackage{fancyvrb}
% \usepackage{cprotect}
\usepackage{lscape}
\usepackage{multirow}

\usepackage{graphicx}
\usepackage{fontspec,xunicode}
\defaultfontfeatures{Mapping=tex-text, Numbers=Monospaced}% ,Scale=MatchLowercase% , Numbers=OldStyle
\setmainfont[Scale=1]{Linux Libertine O}% {Sabon MT Std}
\setsansfont{Roboto Bold}% {Myriad Pro Light}
\setmonofont[Scale=0.82]{Monaco}% {Droid Sans Mono}
\allsectionsfont{\sffamily}

\newfontinstance\bigsf[Color=000000,Scale=1.25]{Roboto Bold}% {Myriad Pro Light}
\newfontinstance\sfapp[Color=000000,Scale=0.82]{Roboto Regular}% {Myriad Pro Light}

\definecolor{red}{rgb}{0.80,0.00,0.00}
\definecolor{green}{rgb}{0.00,0.80,0.00}
\definecolor{blue}{rgb}{0.00,0.00,0.80}

\input{epsf}                            % to allow postscript inclusions
% On thor and CUS read top of file:
%     /opt/TeX/lib/texmf/tex/dvips/epsf.sty
% On CL machines read:
%     /usr/lib/tex/macros/dvips/epsf.tex

\raggedbottom                           % try to avoid widows and orphans
\raggedright                            % left-aligned, not justified
\sloppy
\clubpenalty1000%
\widowpenalty1000%

\addtolength{\oddsidemargin}{6mm}       % adjust margins
\addtolength{\evensidemargin}{-8mm}

\renewcommand{\baselinestretch}{1.1}    % adjust line spacing to make
                                        % more readable

\renewcommand{\captionfont}{\footnotesize}

\setcounter{secnumdepth}{5}
% \setcounter{tocdepth}{5}

\begin{document}
\nocite{*}
\bibliographystyle{plain}% {unsrt}


%%%%%%%%%%%%%%%%%%%%%%%%%%%%%%%%%%%%%%%%%%%%%%%%%%%%%%%%%%%%%%%%%%%%%%%%
% Title


\pagestyle{empty}

% \hfill{\LARGE \bf \sffamily Philip Yeeles}
\hfill{\Large Philip Yeeles}

\vspace*{60mm}
\begin{center}
\LARGE
{\bf \bigsf Nico: An Environment for Mathematical Expression in Schools}\\
\vspace*{5mm}
% {\sffamily Computer Science Tripos} \\
Computer Science Tripos\\
\vspace*{5mm}
% {\sffamily Selwyn College} \\
Selwyn College\\
\vspace*{5mm}
% {\sffamily \today}  % today's date
\today % today's date
\end{center}

\cleardoublepage

%%%%%%%%%%%%%%%%%%%%%%%%%%%%%%%%%%%%%%%%%%%%%%%%%%%%%%%%%%%%%%%%%%%%%%%%%%%%%%
% Proforma, table of contents and list of figures

\setcounter{page}{1}
\pagenumbering{roman}
\pagestyle{plain}

\chapter*{Proforma}

{\large
\begin{tabular}{ll}
\bf Name:               & Philip Yeeles                                                   \\
\bf College:            & Selwyn College                                                  \\
\bf Project Title:      & Nico: An Environment for Mathematical\\
                        & Expression in Schools \\
\bf Examination:        & Computer Science Tripos, May 2012                               \\
\bf Word Count:         & 11,988\footnotemark[1]                                          \\
\bf Project Originator: & P.~M.~Yeeles (\verb¬pmy22¬)                                    \\
\bf Supervisors:        & Dr S.~J.~Aaron (\verb¬sja55¬), A.~G.~Stead (\verb¬ags46¬)    \\
\end{tabular}
}
\footnotetext[1]{This word count was computed using the web interface to {\ttfamily texcount}, which can be found at \url{http://app.uio.no/ifi/texcount/online.php}.}
\stepcounter{footnote}


\section*{Original Aims of the Project}

The aim of this project was to develop an application to act as an aid in mathematical education for pupils in Year 5.  The software was to provide a new graphical metaphor for calculation with an accompanying working environment, with the intention of eliminating some of the problems that makes handwritten arithmetic unclear in many situations.

%The aim of the project was to develop an application in the Clojure programming
%language which would allow users to express mathematical calculations using a
%graphical notation.  The software was to be able to generate an abstract syntax
%tree from the graphical notation, evaluate it and pass the results back to the
%application in under 300ms.  An extension to the project was to conduct a user
%study to evaluate the utility of the software.

\section*{Work Completed}

I have successfully designed and implemented an application using the Clojure programming language that allows users to express calculations using a graphical notation.  The software is able to generate an abstract syntax tree from the graphical notation, evaluate it and pass the results back to the application in approximately 10ms.  As an extension to the project, a user study was conducted to evaluate the utility of the software.

%I have successfully designed and implemented the application detailed in the
%previous section.  That is, I have developed an application in which it is
%possible to express calculations using a graphical notation, that generates an
%abstract syntax tree from the language and that is able to parse the tree and
%return the results in under 300ms.  I have also conducted a user study to assess
%whether or not the software is actually of use with regard to mathematics
%education.

\section*{Special Difficulties}

None.

\newpage
\section*{Declaration of Originality}

I, Philip Michael Yeeles of Selwyn College, being a candidate for Part
II of the Computer Science Tripos, hereby declare that this dissertation
and the work described in it are my own work, unaided except as may be
specified below, and that the dissertation does not contain material
that has already been used to any substantial extent for a comparable
purpose.

\bigskip
\leftline{Signed}

\medskip
\leftline{Date \today}

\cleardoublepage

\tableofcontents

\listoffigures

\newpage
\section*{Acknowledgements}
My thanks to Luke Church for his advice regarding user studies, to Alexandra Hancock and to my supervisors Alistair Stead and Sam Aaron for their encouragement and patience.

%%%%%%%%%%%%%%%%%%%%%%%%%%%%%%%%%%%%%%%%%%%%%%%%%%%%%%%%%%%%%%%%%%%%%%%
% now for the chapters

\cleardoublepage        % just to make sure before the page numbering
                        % is changed

\setcounter{page}{1}
\pagenumbering{arabic}
\pagestyle{headings}

\chapter{Introduction}
% TODO: sort out cogdims: apparently the proper way to do it is to mention cogdims and cite petra & green, then put all the cogdims terms in italics

The aim of this project has been to design and develop a notation and accompanying application to act as a learning aid for pre-algebra arithmetic in schools.  Using Petre and Green's `Cognitive Dimensions' framework as a means of evaluation, the project aims to increase the \emph{visibility} of the notation, to reduce the number of \emph{hidden dependencies} between units of notation, to reduce the \emph{viscosity} of the notation, and to make the flow of data obvious to the user \cite{Green1996}.  I have successfully developed such a system, intended initially for pupils in Year 5 (though extensible, through the creation of alternative question sets, to other age groups), and, as an extension, conducted a user study to assess its utility.

In this chapter, I will discuss my motivations for choosing this project, the pros and cons of the handwritten system it is attempting to augment, the technical challenges involved in developing such a system and related work that has previously been conducted with similar goals.

\section{Motivations}
% evaluation of problems with writing it down on paper
% cogdims of handwritten method(s)
% system requirements (cogdims)

My motivations behind this project lay in the limitations of the handwritten approach to solving mathematical problems that I have observed both in my own learning and in my own teaching experience.  What follows is an assessment of the technical challenges involved in this project, followed by a discussion of the ideal properties of a useful alternative notation.

\pagebreak

% \subsection{Evaluation of the Handwritten Approach}
% % moved to chapter 2 requirements analysis

% \subsection{Requirements of a Replacement System}
% % moved to chapter 2 requirements analysis

\section{Technical Challenges}
% need first-class functions
% need to be able to pass expressions around to evaluate at will: hence functional, hence lisp - homoiconicity
% also need good gui libraries for app - familiar with java/swing so that works (esp. with seesaw), also the option of jfx2 and swt/guiftw
% difficulty of learning clojure over summer in preparation

Developing such an application comprises three main challenges.  Firstly, developing an infrastructure that is capable of creating, storing, editing, deleting, evaluating and nesting calculations.  Secondly, developing an interaction-handler layer that is able to interpret user input and events occurring in the user interface.  Finally, implementing a GUI that is able to render calculations into the devised notation, and allow the user to perform operations upon the notation that affect the underlying calculation.  Such an interface should also be able to display and update the results of a calculation with low latency (a limit of 300ms is set and discussed in footnote 1, \emph{Sec. 2.1.2}).

The application described above must be portable, to account for the diversity of computing environments available to the target audience.  It must also provide a standardised user experience across the platforms on which it is deployed.  A Java-based application satisfies both of these requirements, as any platform on which the Java Virtual Machine (JVM) can be installed is able to run it.  Java also makes GUI libraries available, such as JavaFX and Swing, that are explicitly designed to provide similar GUI functionality on many platforms.

The application would also benefit from functional programming paradigms such as immutability, which results in functions that behave predictably and are easy to test; and homoiconicity, in which data structures can be evaluated as code and code as data structures, allowing mathematical method to be represented in a form that can be interpreted and manipulated by the application.

The Clojure language meets both of the above requirements, being a dialect of LISP that compiles to JVM bytecode.  This gives the developer access to the Java standard library and offers the portability afforded by Java, whilst also providing the benefits of a homoiconic, functional language as described above.  It also has the added benefit of being efficient, as it compiles directly to JVM bytecode.

\section{Previous Work}
% scrubbing calculator http://worrydream.com/ScrubbingCalculator/
% soulver http://www.acqualia.com/soulver/
% neither really targeted at education
% cool stuff at http://worrydream.com/KillMath/ though
% maybe refer to this? http://betterexplained.com/articles/rethinking-arithmetic-a-visual-guide/
% also, this: http://www.ralph-abraham.org/articles/Blurbs/blurb126.shtml
% also, check out espresso and education city stuff, log in to fronter via st. mary's site, uname jyeele1.314, pword b3ach7
% plenty of work been done in making computer maths like paper maths, but not so much the reverse, case in point: http://www.macresearch.org/showcase-review-pi-cubed-iphone-ipod-touch
% also a lot of crap maths games that are essentially a series of qs in disguise, see http://www.time4learning.com/curriculum/try_demos.html
% nico is unabashedly *not* a game, rather a tool, much like a pen or calculator
% problem with calculators?

There already exists a wide variety of educational software for mathematics, which often present mathematical problems in the context of a computer game.  Although such games have repeatedly been shown to be of benefit to the learner, a barrier to their widespread use in education can be the users' preconceptions about the educational value of computer games \cite{Swan2009} \cite{Delacruz2010}.  Indeed, Bragg notes that
\begin{center}
\parbox[c]{\textwidth-2cm}{
\small
``One barrier [...] may be possible negative attitudes held by students towards the likelihood of the games' effectiveness in assisting mathematical learning.'' \cite{Bragg2006}
}
\end{center}
This is reinforced in a later paper by the same author, where Bragg concludes that
\begin{center}
\parbox[c]{\textwidth-2cm}{
\small
``It appeared that game-playing negatively affected attitudes'' \cite{Bragg2007}
}
\end{center}
Although it was also concluded that such problems could be addressed by dedicating lesson time to explaining the educational benefits of game-playing, an alternative approach is to introduce software into the classroom as a {\bf tool}, rather than `just a game'.  In the same way that calculators are used in mathematics lessons, it is the intention of this project to introduce a new tool to complement learning.

There also exist a few applications intended to represent calculations on a computer in novel ways.  A relatively common approach to this has been to try to make on-screen calculations more like on-paper calculations.  One such example is \emph{Pi Cubed}, which tries to make complex calculations appear as they would be written in an exam or exercise book \cite{PiCubed}.  \emph{Soulver}, conversely, tries to achieve this by simulating `back-of-the-envelope' calculations, whereby notes in English augment the calculation \cite{Soulver}.

% TODO: image?  but copyright issues...
Another approach is that of the \emph{Scrubbing Calculator}, which extends the \emph{Soulver}-style environment by helping the user to solve equations by dragging values to edit them, showing how changing a value affects the overall result \cite{ScrubCalc}.  Values can be linked by dragging a line between them, which means that they are two instances of the same value: dragging one changes the value at every location in which it appears.  This is a neat means of visualising equations, but it, too, is not intended for use in education, and still requires the user to be able to formulate an equation.  The \emph{Scrubbing Calculator} is more a tool for facilitating algebraic, rather than arithmetic, understanding; indeed, it is inherently a {\bf calculator}, and so does not encourage thinking about a method of manual calculation.

\section{Summary}
% need a system that incorporates these features: foo, bar, baz, quux

Existing educational games for mathematics, although valid as a pedagogical tool, face obstacles from teachers' and pupils' attitudes toward them, as a result of their status as `just games'.  There exists software to aid in calculation and arithmetic by representing it clearly, but it is not intended for classroom use, and often its purpose is to make on-screen calculations simulate handwritten ones.  An application is proposed to complement teaching, as one would use a calculator in the classroom.  There is a niche for a tool for use in education that represents calculations in a visual manner, with a particular focus on making mathematical {\bf method} clear.  The application is to provide such an environment, in which the user can explore the many ways in which a problem can be solved using a novel graphical notation.% TODO: emphasise? wtf

\cleardoublepage



\chapter{Preparation}

% In this chapter I will discuss the work that was done prior to beginning the
% project proper.  This includes learning the Clojure programming language and
% researching and auditioning graphical metaphors for calculation.  This chapter
% comprises a requirements analysis, followed by an overview of the system
% architecture and a discussion of the additional tools used in the development of
% the project.
This chapter concerns the work that was completed prior to development.  It comprises a requirements analysis, followed by a discussion of the prototyping of the graphical notation that was to be implemented in the final application.  A list of third-party tools used can be found in \emph{Appendix A}.

\section{Requirements Analysis}
% http://www.cs.fsu.edu/~lacher/courses/COP3331/rad.html
% http://www.nd.gov/itd/files/services/pm/requirements-analysis-guidebook.pdf
% TODO: write about the requirements analysis and why we're doing one

\subsection{Current System}
% move evaluation of handwritten arithmetic from chapter 1 here?

\begin{figure}
% \begin{wrapfigure}{r}{0.5\textwidth}
\begin{center}
\subfloat[]{
\parbox{2cm}{
\begin{center}
{\small
24+35=59\\
12+48=60\\
59+72=131\\
1+60=61\\
131+61=192}
\end{center}}}
\subfloat[]{
\parbox{2cm}{
\begin{center}
{\small
24+35={\color{green}59}\\
12+48={\color{blue}60}\\
{\color{green}59}+72={\color{green}131}\\
1+{\color{blue}60}={\color{blue}61}\\
{\color{green}131}+{\color{blue}61}=192}
\end{center}}}
\subfloat[]{
\parbox{4cm}{
\begin{center}
{\small
24+35= {\color{red}\sout{59}} {\color{green}49}\\
12+48={\color{blue}60}\\
{\color{red}\sout{59}} {\color{green}49}+72= {\color{red}\sout{131}} {\color{green}121}\\
1+{\color{blue}60}={\color{blue}61}\\
{\color{red}\sout{131}} {\color{green}121}+{\color{blue}61}= {\color{red}\sout{192}} 182}
\end{center}}}
\end{center}
\caption{Illustrating the hidden dependencies and viscosity inherent in pre-algebra, handwritten arithmetic.  The original calculation is shown in \emph{a)}; in \emph{b)}, it has its otherwise-hidden dependencies highlighted; and it is altered slightly in \emph{c)}.}
\label{fig:ArithHDs}
% \end{wrapfigure}
\end{figure}
% cogdims
% taught method without being taught *why* it works
% high viscosity - have to cross things out or start again
% high repetition viscosity
% low knock-on viscosity
% viscosity is acceptable in transcription and incrementation, but harmful to modification and exploration
% hidden dependencies -
% premature commitment - made worse by high repetition viscosity; have to start again if a mistake is made
% premature commitment is certainly very high for traditional calculators
% algebra is abstraction-hungry, arthimetic is abstraction-hating, though can use secondary notation to construct abstractions
% no abstraction barrier to arithmetic though
% secondary notation - notes around calculations
% visibility - visible, but only juxtaposable if juxtaposed at the start (premature commitment) or with the help of secondary notation
There are a number of problems with handwritten, pre-algebra arithmetic that this project seeks to rectify.  This section includes an analysis of the handwritten method according to the `Cognitive Dimensions' framework.

\subsubsection{Disadvantages}
Firstly, the fact that the traditional notation for arithmetic is handwritten entails a high level of \emph{viscosity}: it is difficult to make changes to a calculation without sacrificing clarity.  Modification entails \emph{repetition viscosity}: if a number is changed that is reused, it is time-consuming to change its every appearaance in the working.  \emph{Knock-on viscosity} becomes problematic where there are dependencies between calculations, requiring recalculation at each stage after modification.  This is exacerbated by \emph{hidden dependencies} between chained calculations (\emph{Fig.~\ref{fig:ArithHDs}}).

\emph{Hidden dependencies} increase with the low \emph{juxtaposability} of the notation.  Although the notation is quite \emph{visible}, in that every calculation can be seen easily on the page, juxtaposing two sets of calculations requires \emph{premature commitment}, whereby decisions regarding component placement are made before appropriate solutions can be deduced.  This is compunded by the system's high \emph{viscosity}.

Whilst \emph{viscosity} can be acceptable in some situations, it is harmful with regard to modification and exploration within a notational system, requiring a non-trivial amount of \emph{premature commitment}.  It can be quicker for the user to start over again, as opposed to amending their calculations.

\subsubsection{Abstraction Barrier}
Handwritten arithmetic does have one key advantage: it has a low initial \emph{abstraction barrier}.  Once the symbols and rules for each operation and digit have been learnt, this system allows a learner to begin using it almost immediately.  Although it is possible to add abstractions to arithmetic by the use of secondary notation, there is no provision for abstraction included in the primary.  Algebra, for example, is an \emph{abstraction-hungry} system, requiring the concept of a variable rather than a set quantity; whereas arithmetic is an \emph{abstraction-hating} system.  However, without abstractions, arithmetic can become very verbose and \emph{viscous}, with low \emph{visibility}.

To improve upon the standard approach of listing the steps comprising a calculation, a notational system must overcome the drawbacks listed above.  To this end, I have designed and developed such a system and accompanying application, intending to reduce \emph{viscosity}, increase \emph{visibility} and remove many of the \emph{hidden dependencies} that beleaguer arithmetic.  The functional and non-functional requirements of this system follow.

%\subsection{Proposed System}
%
%\subsubsection{Overview}
%% cogdims!!
%% need to develop useful notation - early studies, talk about other notations we came up with, notations looked at with alistair (get link)
%% hidden dependencies - nico shows a data flow representation (p19, blackwell1998) to make dependencies explicit
%% TODO: this bit.  not even sure if it should be in the intro...
%
%The proposed system is a novel graphical notation
%
\subsection{Functional Requirements}

The new system must satisfy the following functional requirements to be considered acceptable.  It must be able to:

\begin{itemize}
\item Present the user with a series of problems to solve
\item Accept a file containing the set of questions to be answered
\item Display the current question being answered
\item Determine whether or not the user's solution is correct
\item Progress through the current question set as the user answers each question correctly
\item Interpret the user's work and display the result interactively \footnote{Roca and Rousseau note that: ``An abundance of studies into user tolerance of round-trip latency [...] has been conducted and generally agrees upon the following levels of tolerance: excellent, 0-300ms; good, 300-600ms; poor, 600-700ms; and quality becomes unacceptable [...] in excess of 700ms.'' \cite{Roca2004}}
\end{itemize}
%To be an improvement upon the current system detailed above, the new system must
%satisfy the following properties:-- % TODO: not happy with this
%\begin{itemize}
%\item Allows the user to create graphical structures representative of complex calculations
%\item Provides the user with a suitable environment in which to do so
%\item Allows the user to reposition elements of the structure at will
%\item Displays the user's current progress
%\item Is able to evaluate the correctness of the user's answer
%\item Accepts a file containing a set of questions to be answered
%\item Displays the current question being answered
%\item Progresses through the current question set as the user answers each question correctly
%\end{itemize}

\subsection{Non-Functional Requirements}

The system must also satisfy a number of non-functional requirements.  It must, compared to handwritten arithmetic:

\begin{itemize}
\item Offer an improvement in \emph{visibility}
\item Offer an improvement in \emph{juxtaposability}
\item Reduce \emph{premature commitment}
\item Reduce \emph{hidden dependencies}
\end{itemize}

And also:

\begin{itemize}
\item Have a visually simple interface whilst remaining accessible to older users
\end{itemize}

\section{User Interface}
% section rather than subsection?  def. section if we don't have system overview
% prototypes!  dig out the preliminary drawings, take screenshots of old versions for later

As this project is primarily concerned with human-computer interaction, the user interface and experience is significant.  As such, this section outlines the initial development stages of the UI, first introducing several designs for the graphical notation, then proceeding to discuss in more detail three examples that were developed further.

\subsection{Prototyping}

\subsubsection{Low-Fidelity Prototyping}

% drawings 'n' shit
% talk about how decided to come up with 20 different designs then thin them out
To devise initial ideas for what could become the calculation metaphor, a target was set of devising at least twenty significantly-different potential designs.  These were recorded as rough sketches, three of which were warranted an application mockup, as detailed in \emph{Sec.~2.2.1.2}.

\begin{center}
\begin{figure}[H]
% \begin{wrapfigure}{r}{0.5\textwidth}
\begin{center}
\subfloat[]{
\parbox{4cm}{
\begin{center}
\includegraphics[width=4cm]{figs/mockups/sketches/11/11a.jpg}
\end{center}}}
\subfloat[]{
\parbox{4cm}{
\begin{center}
\includegraphics[width=4cm]{figs/mockups/sketches/11/11b.jpg}
\end{center}}}
\end{center}
\caption{Early designs for flowgraph-based calculation metaphors.}
\label{fig:Flows1}
% \end{wrapfigure}
\end{figure}
\end{center}

The designs in \emph{Fig.~\ref{fig:Flows1}} use a flowgraph-based metaphor, in which variables and functions are represented as nodes in the graph.  Dataflow is extremely clear in such diagrams, giving these notations high \emph{visibility}.  These designs are similar to the one in the project proposal (\emph{Appendix D}).  The boxes and arrows occupy relatively little screen-estate, which is advantageous in cases where a question necessitates many subcalculations.  \emph{Fig.~\ref{fig:Flows2}} shows a refined version of a flowgraph notation, with an environment in which to utilise it.

\begin{center}
\begin{figure}[H]
\begin{center}
\includegraphics[width=\textwidth]{figs/mockups/sketches/11/11jii.jpg}
\end{center}
\caption{A refined sketch of a flowgraph-based language and application.}
\label{fig:Flows2}
\end{figure}
\end{center}

\begin{center}
\begin{figure}[H]
% \begin{wrapfigure}{r}{0.5\textwidth}
\begin{center}
\subfloat[]{
\parbox{4cm}{
\begin{center}
\includegraphics[width=4cm]{figs/mockups/sketches/21/21b.jpg}
\end{center}}}
\subfloat[]{
\parbox{4cm}{
\begin{center}
\includegraphics[width=4cm]{figs/mockups/sketches/22/22a.jpg}
\end{center}}}
\subfloat[]{
\parbox{4cm}{
\begin{center}
\includegraphics[width=4cm]{figs/mockups/sketches/22/22b.jpg}
\end{center}}}\\
\subfloat[]{
\parbox{4cm}{
\begin{center}
\includegraphics[width=4cm]{figs/mockups/sketches/42/42a.jpg}
\end{center}}}
\subfloat[]{
\parbox{4cm}{
\begin{center}
\includegraphics[width=4cm]{figs/mockups/sketches/42/42b.jpg}
\end{center}}}
\end{center}
\caption{Early designs for triangle-based calculation metaphors.}
\label{fig:Tris1}
% \end{wrapfigure}
\end{figure}
\end{center}

Triangle-based notations were also considered, drawing inspiration from the familiar forms of the formula triangles used in science education (\emph{Fig.~\ref{fig:Tris1}}, \emph{c)} and \emph{d)}) and from the Sierpiński triangle fractal (\emph{Fig.~\ref{fig:Flows2}}, \emph{a)} and \emph{e)}).  Such notations have high \emph{visibility}.  Their \emph{juxtaposability}, however, is low, as a rigid structure is imposed upon calculations.  A refined sketch is shown in \emph{Fig.~\ref{fig:Tris2}}.

\begin{center}
\begin{figure}[H]
\begin{center}
\includegraphics[width=\textwidth]{figs/mockups/sketches/31/31a.jpg}
\end{center}
\caption{A refined sketch of a triangle-based language.}
\label{fig:Tris2}
\end{figure}
\end{center}

\newpage
\begin{center}
\begin{figure}[H]
% \begin{wrapfigure}{r}{0.5\textwidth}
\begin{center}
\subfloat[]{
\parbox{4cm}{
\begin{center}
\includegraphics[width=4cm]{figs/mockups/sketches/12/12a.jpg}
\end{center}}}
\subfloat[]{
\parbox{4cm}{
\begin{center}
\includegraphics[width=4cm]{figs/mockups/sketches/12/12b.jpg}
\end{center}}}
\subfloat[]{
\parbox{4cm}{
\begin{center}
\includegraphics[width=4cm]{figs/mockups/sketches/12/12c.jpg}
\end{center}}}\\
\subfloat[]{
\parbox{4cm}{
\begin{center}
\includegraphics[width=4cm]{figs/mockups/sketches/12/12f.jpg}
\end{center}}}
\subfloat[]{
\parbox{4cm}{
\begin{center}
\includegraphics[width=4cm]{figs/mockups/sketches/12/12h.jpg}
\end{center}}}
\subfloat[]{
\parbox{4cm}{
\begin{center}
\includegraphics[width=4cm]{figs/mockups/sketches/41/41a.jpg}
\end{center}}}\\
\subfloat[]{
\parbox{4cm}{
\begin{center}
\includegraphics[width=4cm]{figs/mockups/sketches/41/41b.jpg}
\end{center}}}
\end{center}
\caption{Early designs for circle-based calculation metaphors.}
\label{fig:Circs1}
% \end{wrapfigure}
\end{figure}
\end{center}

Several circle-based notations were also considered, the focus here being on subexpressions as whole units that can easily be relocated, providing high \emph{juxtaposability}. This is opposed to using separate numbers and operations as the basic units of a large calculation.  These also offer an improvement in \emph{visibility} over handwritten arithmetic, as discussed in \emph{Sec.~2.2.1.2.3}.  A detailed sketch is shown below.

\begin{center}
\begin{figure}[H]
\begin{center}
\includegraphics[width=\textwidth]{figs/mockups/sketches/21/21a.jpg}
\end{center}
\caption{A refined sketch of a circle-based language.}
\label{fig:Circs2}
\end{figure}
\end{center}

% \newpage

% \paragraph{Other}\hfill

\begin{center}
\begin{figure}[H]
% \begin{wrapfigure}{r}{0.5\textwidth}
\begin{center}
\subfloat[]{
\parbox{4cm}{
\begin{center}
\includegraphics[width=4cm]{figs/mockups/sketches/11/11c.jpg}
\end{center}}}
\subfloat[]{
\parbox{4cm}{
\begin{center}
\includegraphics[width=4cm]{figs/mockups/sketches/11/11d.jpg}
\end{center}}}
\subfloat[]{
\parbox{4cm}{
\begin{center}
\includegraphics[width=4cm]{figs/mockups/sketches/11/11e.jpg}
\end{center}}}\\
\subfloat[]{
\parbox{4cm}{
\begin{center}
\includegraphics[width=4cm]{figs/mockups/sketches/11/11f.jpg}
\end{center}}}
\subfloat[]{
\parbox{4cm}{
\begin{center}
\includegraphics[width=4cm]{figs/mockups/sketches/11/11g.jpg}
\end{center}}}
\subfloat[]{
\parbox{4cm}{
\begin{center}
\includegraphics[width=4cm]{figs/mockups/sketches/11/11h.jpg}
\end{center}}}\\
%\end{center}
%\end{figure}
%\newpage
%\begin{figure}
%\ContinuedFloat
%\begin{center}
\subfloat[]{
\parbox{4cm}{
\begin{center}
\includegraphics[width=4cm]{figs/mockups/sketches/11/11i.jpg}
\end{center}}}
\subfloat[]{
\parbox{4cm}{
\begin{center}
\includegraphics[width=4cm]{figs/mockups/sketches/11/11k.jpg}
\end{center}}}
\subfloat[]{
\parbox{4cm}{
\begin{center}
\includegraphics[width=4cm]{figs/mockups/sketches/11/11li.jpg}
\end{center}}}\\
\subfloat[]{
\parbox{4cm}{
\begin{center}
\includegraphics[width=4cm]{figs/mockups/sketches/11/11lii.jpg}
\end{center}}}
\subfloat[]{
\parbox{4cm}{
\begin{center}
\includegraphics[width=4cm]{figs/mockups/sketches/11/11m.jpg}
\end{center}}}
\subfloat[]{
\parbox{4cm}{
\begin{center}
\includegraphics[width=4cm]{figs/mockups/sketches/12/12d.jpg}
\end{center}}}\\
\subfloat[]{
\parbox{4cm}{
\begin{center}
\includegraphics[width=4cm]{figs/mockups/sketches/12/12e.jpg}
\end{center}}}
\end{center}
\caption{Assorted early designs for other calculation metaphors.}
\label{fig:Others1}
% \end{wrapfigure}
\end{figure}
\end{center}

\begin{center}
\begin{figure}[H]
\begin{center}
\subfloat[]{
\parbox{4cm}{
\begin{center}
\includegraphics[width=4cm]{figs/mockups/sketches/12/12gi.jpg}
\end{center}}}
{\Huge $ \rightarrow $}% arrow
\subfloat[]{
\parbox{4cm}{
\begin{center}
\includegraphics[width=4cm]{figs/mockups/sketches/12/12gii.jpg}
\end{center}}}
{\Huge $ \rightarrow $}\\% arrow
\subfloat[]{
\parbox{4cm}{
\begin{center}
\includegraphics[width=4cm]{figs/mockups/sketches/12/12giii.jpg}
\end{center}}}
{\Huge $ \rightarrow $}% arrow
\subfloat[]{
\parbox{4cm}{
\begin{center}
\includegraphics[width=4cm]{figs/mockups/sketches/12/12giv.jpg}
\end{center}}}
{\Huge $ \rightarrow $}% arrow
\subfloat[]{
\parbox{4cm}{
\begin{center}
\includegraphics[width=4cm]{figs/mockups/sketches/12/12gv.jpg}
\end{center}}}
\end{center}
\caption{A sample progression of a calculation using a language based around an infinitely-subdivided rectangle.}
\label{fig:Others2}
\end{figure}
\end{center}

Other metaphors were also considered, as illustrated in \emph{Figs.~\ref{fig:Others1} and \ref{fig:Others2}}.  \emph{Fig.~\ref{fig:Others1} a)} was considered too similar to Google/MIT's \emph{App Inventor}.  Sketches \emph{c)}, \emph{f)}, \emph{l)} and \emph{m)} were deemed too cumbersome, and \emph{d)}, \emph{e)} and \emph{h)} to \emph{k)} were not considered visually appealing, appearing too similar to existing mathematical notations and constructs.  Items \emph{b)} and \emph{g)} were inflexible, and would have scaled poorly to larger calculations.  Finally \emph{Fig.~\ref{fig:Others2}} was rejected as it was too similar to the Sierpiński-triangle metaphor shown in \emph{Fig.~\ref{fig:Tris2}}.

\subsubsection{Detailed Mockups}
% examples done using that prototyping software - balsamiq mockups
% draft designs on pwf

More mature designs for the application and notation follow.  First is a combined flowgraph representation and Read-Evaluate-Print-Loop (REPL) design.  The second is based around the Sierpiński-triangle metaphor.  The final design is a circle-based language, similar to \emph{Fig.~\ref{fig:Circs2}}.

\newpage
% TODO: these fuckers
\paragraph{Flowgraph Metaphor}\hfill

\begin{figure}[H]
\begin{center}
\includegraphics[width=\textwidth-4cm]{figs/mockups/flowgraph/img/fg-000.jpg}
\caption{The prototype flowgraph-style language and accompanying application.}
\label{fig:ProtoFlow}
\end{center}
\end{figure}

This layout expands the application prototyped in the project proposal.  The application now comprises one window containing information panels, a `toolbox' of available functions that can be dragged on to the canvas, and a canvas area upon which calculations are constructed.  Zoom and help buttons are provided.  The user is able to use the {\sfapp Flow} and {\sfapp Worksheet} tabs to switch between the graphical mode and the REPL mode respectively.

The top screen in \emph{Fig.~\ref{fig:ProtoFlow}} shows manipulation of the graphical language, whilst the bottom screen shows a REPL utilising a simple text-based language.  The flowgraph notation was used as it is \emph{visible}, making the flow of data very clear to the user and allowing them to see much of their structure at once.  The notation also occupies little screen-estate, allowing it to accommodate large calculations.

In the original design, no control scheme was specified; the user was expected to have control over many different functions, meaning that without a simple control scheme, the user may require guidance.  The toolbox was included to exhibit available functions.  It includes two buttons: {\sfapp Join}, which changes the function of the mouse to join boxes with arrows; and {\sfapp Define}, which similarly allows the user to draw a box around a section of the diagram to define a function.  This was introduced to allow for the abstraction of repeated tasks, making the notation more compact.  It was decided to use these modes, combined with a default mode allowing relocation of boxes, so that the control scheme could be implemented using one mouse button.

Zoom buttons were included to accommodate large diagrams, and to allow the user to view their work at a comfortable size.  The help button was included to provide assistance for new users.  Activating help would display a user manual.

{\sfapp Worksheet} mode was included for users who may feel more comfortable in a textual environment, rather than using the graphical metaphor.

This design was abandoned as it was deemed to be too complex for the target audience of Year 5.  The intention of this project was not to implement an educational programming language, and by including features such as the REPL and {\sfapp Define}, the system overprovides.  The REPL is superfluous, as users preferring text are likely to be comfortable using the handwritten method.

In addition to this, the flowgraph representation had a number of more general shortcomings. For example, the evaluation order of arguments is unclear, which is confusing for non-commutative functions such as subtraction.  Furthermore, the language is verbose.  This is not inherently a disadvantage but a more compact representation would make expressions more \emph{visible}, and decrease time spent by the user searching for information within the notation.  It is also \emph{viscous} when replacing whole subcalculations, rather than just numbers or operators, though less so than handwritten arithmetic.

\newpage
\paragraph{Sierpiński Triangle Metaphor}\hfill

\begin{figure}[H]
\begin{center}
\includegraphics[width=\textwidth]{figs/mockups/sierp/sierp_mockup_full.png}
\caption{The prototype Sierpiński-triangle-based language and accompanying application.}
\label{fig:ProtoTri}
\end{center}
\end{figure}

This prototype uses a zooming instance of the Sierpiński-triangle fractal as the basis of its notation (\emph{Fig.~\ref{fig:ProtoTri}}).  The application comprises a single window with information panels and a canvas area, featuring zoom and help buttons, as well as control buttons for diagram construction.

This prototype marks a significant departure from the flowgraph metaphor, to address its many problems.  The application also had to change to suit the new notation: in this case, a toolbox would not be suitable, as the user is limited to one structure.

The triangle notation is more \emph{visible} than the handwritten method and emphasises structure, clarifying the way smaller calculations combine to form larger ones by using a fractal-based notation.  It is colour-coded by each triangle's level of nesting, showing at which level the user is working and allowing the user to abstract away smaller triangles as arguments to encompassing calculations.

Zoom buttons were included as, although the size of the overall structure always remains the same in this notation, it becomes necessary to read very small sections as nested calculations are constructed.  Indeed, each section constitutes a quarter of the area of the triangle that contains it, as illustrated in \emph{Fig.~\ref{fig:ProtoTri}}.  As in the previous prototype, a help button was included to ease initiation.

A control scheme using buttons was again chosen so that only one mouse-button was required.  The arrangement of these buttons maps to the arrangement of triangles within the diagram.  This was to indicate the consequences of each control upon the structure: the {\sfapp Input} and {\sfapp Op} buttons allow the user to modify the corresponding sections of the triangle.  Answers are submitted using the {\sfapp Answer} button, allowing them to be checked and changed before submission.

This design was ultimately abandoned as it was awkward. It is based around binary operations, meaning that to solve the simple question 1+2+3+4, one has to either calculate ((1+2)+3)+4, which is inefficient; or deduce that 1+2+3=6 and then calculate 6+4, which is unacceptable for a system designed to aid arithmetic learning.

The severe lack of \emph{juxtaposability} is also problematic: as each expression must fit into the larger structure, it is not possible to reorder them without fundamentally altering the calculation being performed.  It also enforces a top-down approach to calculation: one is required to begin with the outermost calculation and work inwards.  There is no provision for encapsulating an existing calculation within another triangle without significant \emph{premature commitment}.

In addition to this, an informal survey regarding potential notations was conducted, asking approximately ten participants to answer some sample questions in this notation and the circle-based notation below.  Participants found this notation less clear, and were able to complete more questions correctly using the circle notation.
% survey
% TODO: can't put subexpressions into right-hand triangle
%
% what/mockup
% application
% language
% abandoned

\newpage
\paragraph{Circle Metaphor}\hfill

\begin{figure}[H]
\begin{center}
\includegraphics[width=\textwidth]{figs/mockups/circles/nico_circmock.png}
\caption{The prototype circle-based language and accompanying application.}
\label{fig:ProtoCirc}
\end{center}
\end{figure}

In this prototype, circles constitute individual units of calculation, linked to indicate the flow of information. Each circle consists of an operator, around which arguments orbit, which are evaluated clockwise from 0°.  It is related to the flowgraph metaphor above, but has a number of key differences.

The application comprises a single window with information panels and a canvas, which includes control, zoom and help buttons, similar to the previous two models.  Zoom buttons accomodate large diagrams, again allowing the user to focus on small sections and to work at a scale that is comfortable.  The help button eases initiation.

Similar to the {\sfapp Define} and {\sfapp Join} buttons in \emph{Fig.~\ref{fig:ProtoFlow}}, the buttons specify mouse modes: {\sfapp New} causes clicking to create new circles, {\sfapp Split} causes clicked circles to gain an argument, and so on. The motivations for this were similar to that for the modes in the flowgraph model, including the ability to control the software using one mouse button.  The user is able to submit answers using the {\sfapp Answer} button, to allow them to check and change their answers before submission.

Again, this prototype is markedly different to the idea originally proposed.  The application has changed to accommodate the new notation: a toolbox is unsuitable as the notation uses only one fundamental structure, a circle, and the controls in the Sierpiński model (\emph{Fig.~\ref{fig:ProtoTri}}) are inappropriate.

Unlike the Sierpiński metaphor, the user is not restricted to one order or pattern of constructing a calculation.  The circle `tails' make the flow of information through the diagram clear, and are colour-coded by their level of nesting, to further clarify dataflow and to eliminate \emph{hidden dependencies}.

The circle metaphor was chosen for similar reasons to the triangle model: it emphasises the idea of subcalculations as individual units to be abstracted away and treated as any other argument.  This notation has greater \emph{visibility} than handwritten arithmetic, and offers more flexibility than the previous two models, decreasing \emph{viscosity} by allowing the user to edit whole expressions within a calculation and exchange them as needed.  \emph{Juxtaposability} is increased as whole subcalculations are easily repositioned.  The system requires little \emph{premature commitment}, as any piece of the diagram can be modified at will.  This is ideal for exploration, allowing the user to try many ways of solving a problem within the application, rather than having to use a secondary notation or device to help work through a problem and transcribing the result.

This design was ultimately chosen as it solved, or at least ameliorated, many of the problems listed with handwritten arithmetic, without being too complex for the target audience, and without being too restrictive with regard to mathematical expression.  Furthermore, the results of the survey mentioned above also indicated that participants found this notation more intuitive, and made fewer mistakes.

% what/mockup
% application
% language
% chosen

% \section{Additional Tools}

% Luke's:
% Throughout the project, I developed additional support tools to assist with
% design, development and testing. These will be described at the relevant place
% in the implementation chapter.
% lol

\section{Summary}

Prior to the implementation of the project, a thorough requirements analysis was conducted to outline the expected behaviour of the application, contrasting it with the existing system of handwritten calculations.  In \emph{Sec.~2.2}, prototypes for a suitable graphical notation were developed, and three were explored in detail.  Two of these were offered as alternatives to the flawed original design.  The final choice of the circle-based metaphor has many advantanges over the original flowgraph notation.

\cleardoublepage
\chapter{Implementation}

This chapter includes a discussion of the system architecture, detailing the implementation of its infrastructure, management of mutable state, administration of circles, interpretation and evaluation of the notation, and handling of the questions posed to the user.  This is followed by a discussion of the interaction handler, the libraries used, the underlying structure of the user interface and the control scheme.  Next, there is an analysis of the development of the UI, considering the application itself and its control scheme, use of colour-coding and notation.  Finally there is a discussion of testing methods used.

The code itself totals approximately 1,500 lines of Clojure, and was developed in a bottom-up manner, using the iterative and incremental method of software development, in which components of the application are planned, implemented, tested and evaluated before being integrated with larger components.  The application satisfies the success criteria in the project proposal, namely:
\begin{center}
\parbox[c]{\textwidth-2cm}{
\small
``[to be] able to generate an abstract syntax tree in Clojure from the graphical language and evaluate such a tree, passing the results back to the graphical application and displaying this to user in less than 300ms.''
}
\end{center}
In addition to this, an extension to the project was completed, in the form of a user study.

\newpage
\begin{landscape}
\begin{center}
\begin{figure}[H]
\begin{center}
\includegraphics[height=\textheight-2cm]{figs/nico_arch_new.pdf}
\end{center}
\caption{A model of \emph{Nico}'s three-layer architecture.}
\label{fig:NicoArch}
\end{figure}
\end{center}
\end{landscape}

\newpage
\section{System Architecture}
% TODO: intro?

The system comprises three layers of operation: the user interface, the interaction handler and the infrastructure.  Events occurring in the UI are processed by the interaction handler, with the resultant data being passed to the infrastructure.  This handles the interpretation of such data, as well as the memory-management operations.  Data can also flow from the infrastructure to the interaction handler, with results being passed to the interaction handler, causing effects in the UI.  A detailed analysis of each layer follows.
%\begin{center}
%\begin{figure}[H]
%\begin{center}
%\includegraphics[width=\textwidth-2cm]{figs/nico_arch.pdf}
%\end{center}
%\caption{A diagram of workflow in \emph{Nico}.}
%\label{fig:NicoFlow}
%\end{figure}
%\end{center}

% TODO: sort out all the pictures, pdfxetex just can't handle that shit (i.e. scaling images - do it in gimp or something)
% TODO: overview of system architecture beforehand
\subsection{Infrastructure}
% calculations as data structures (lists to be evaled)
% rendering engine
% questions and marking
% question syntax
% questions as data structures (lists to be evaled)
% the fact that they are essentially answers means that answers can easily be compared to them, as well as manipulated and stuff - case in point, all the question-highlighting bullshit
% interactivity (300ms rule, see proposal for citation)
% very little mutable state
% placeholders?  why were they introduced?  how did they make things easier?  what problems did they solve?

% TODO: explain components rather than code, make it higher level
The application infrastructure handles mathematical operations; that is, it is able to interpret data from the user's input and process it as a calculation.

Circles could not simply be represented in the application as S-expression versions of the calculations they stood for, as they are not defined solely by their expressions.\footnote{``S-expression'', meaning ``symbolic expression'', refers to nested list data structures that are used in LISP and its derivatives to represent both source code and data.  S-expressions are defined by McCarthy as a single atom, or as an expression of the form {\ttfamily (a . b)}, where {\ttfamily a} and {\ttfamily b} are S-expressions, thus creating a tree structure \cite{McCarthy1960}.}  Data were needed regarding their position on the canvas, so it was decided to use an associative map to represent a circle, containing co-ordinate data as well as the calculation S-expression.

Originally, the program was to use two macros, \verb¬defcircle¬ and \verb¬letcircle¬, to be able to define circles at runtime.  It became apparent that circles defined in this way were not accessible by the interaction handler whilst the program was running.  To solve this problem, `name' data was added to the circle format, and a mutable data structure was chosen to manage a list of circles currently in use.  This allowed the application to access and modify the circles as needed, and also to discard unused circles.

\subsubsection{Management of Mutable State}

\emph{Nico's} infrastructure stores and manages circles and questions using just six mutable data structures:
\begin{itemize}
\item The list of circles
\item The current question set: a list of maps containing the question number and the S-expression that constitutes the question to be answered
\item The number of circles previously created
\item A map containing the current co-ordinates of the cursor
\item An integer indicating the question currently being answered
\item A map of the circle currently being repositioned and associated data
\end{itemize}

Mutable data structures were chosen as these elements needed to be continually updated whilst the program was running, and having so few is advantageous.  By reducing the number of elements able to change during the execution of the program, the application is made more stable.  The structure of the application is also easier to test and to comprehend: functions called upon immutable objects will not vary in their output, thus those parts of the program that do not interact with the mutable elements behave predictably.

%\begin{center}
%\begin{figure}[H]
%\begin{center}
%\includegraphics[width=\textwidth-2cm]{figs/nico_backend.pdf}
%\end{center}
%% \caption{A rough model of \emph{Nico's} management of mutable state.}
%\caption{A model of \emph{Nico's} management of mutable state; the sixth agent is omitted as it is not utilised by the back-end.}
%\label{fig:Backend}
%\end{figure}
%\end{center}
%
\subsubsection{Circle Management}

Two main functions for the management of circles have been implemented: creation and deletion.\footnote{Editing of circles is accomplished by deleting a circle and recreating a modified version of it.}  New circles are centred on the current co-ordinates of the mouse cursor, with a name of the format \verb¬"c¬\emph{n}\verb¬"¬, where \emph{n} is the number of previously-created circles.  Its expression is initialised to \verb¬(\? \? \?)¬, a meaningless expression using the question-mark character \verb¬\?¬ as a placeholder value for both the operator and the two arguments.\footnote{In this dissertation, the placeholder question mark character is represented as {\ttfamily \textbackslash?} when referring to the question mark character literal in Clojure code, and by {\sfapp ?} when referring to the question mark that is displayed for placeholders in the application.}

Placeholder expressions are used so that the user is able to create a circle without having to specify parameters for it beforehand, which would entail \emph{premature commitment}, as well as the \emph{hard mental operation} of having to visualise a circle before committing to creating it.  Such an approach was exhibited with a previous revision of the software, using a complex dialogue box (\emph{Fig.~\ref{fig:GUDB}}).

It was also decided that the circle should use explicitly {\bf placeholder} values, rather than an expression of value 0 (such as (0+0)) so that new circles did not affect the current total displayed by becoming the new `root circle' (see \emph{Sec.~3.1.1.3}).  The placeholders also indicate that action is needed, whereas an expression of value 0 could appear to be a deliberate inclusion.

Deletion removes a circle from the list by searching through the main circle list and returning a version which lacks the circle matching either the name or the co-ordinates specified.  The circle list is then updated.

This naïve implementation of deletion was problematic: it handles circles that are used as arguments to other circles poorly.  When such a circle is removed from the main list, references to it remain in the circles that used it, causing the application to throw \verb¬java.lang.NullPointerException¬s.  As such, a version of deletion was implemented that preserves referential integrity.

The alternative deletion function works by first checking if any of the circles currently in the main list contain references to the circle being deleted.  If this is the case, the function determines if that circle is a root circle (\emph{Sec.~3.1.1.3}).  If so, the root circle is removed using the previous method and replaced, with invalid references changed to placeholders.  If the circle is non-root, the deletion function recursively calls itself upon the circle, thus removing all circles between the circle being deleted and the root.  The circle can then be deleted as above, and the intermediates are replaced.

The alternative method of deletion maintains referential integrity, but it has also raised a new issue: as this function must call itself upon the other circles in the chain, deleting a circle that forms part of a chain deletes all of the links between the target and the root.  A better way to implement deletion would have been to make the alternative deletion-function inspect the circle list after deletion, replacing any invalid references with \verb¬\?¬s, rather than doing so at delete-time.

\subsubsection{Diagram Evaluation}

There are two main stages required to evaluate the user's work: finding the root circle, and evaluating a circle or collection of circles to give a valid S-expression.

The `root circle' is the one circle in a finished diagram that is not used as an argument to any others.  Therefore, it can be found by searching through the list of circles, comparing each one to the rest of the list and discarding it if it is referred to by other circles.  In a {\bf finished} diagram, there can only be one root circle, so this method is sufficient for marking answers.

It is possible that the user may construct two or more diagrams in parallel, intending to join them later.  This results in multiple roots.  The method chosen to identify the root returns the first root circle that it encounters.  It would have been possible to, for example, add an extra answer display for each root and to maintain a total for each, but it was decided that the user may find changes in the structure of the interface confusing.  This would also require reintroducing a labelling system for circles, which was abandoned as an unfamiliar concept to non-programmers and those without algebra (the majority of the target audience).

Once the root has been found, the user's diagram can then be evaluated to give an S-expression.  As each circle contains an S-expression for the calculation unit that it represents, the references to other circles can be resolved, creating a nested S-expression that can have Clojure's \verb¬eval¬ called on it to return an answer.

This was, in fact, a significant reason for choosing Clojure; as a homoiconic language, code to be evaluated later can easily be built up as a data structure, which can then be passed around, evaluated and otherwise utilised.  By creating one S-expression representing the entire diagram, not only can the value of the structure be calculated, but other operations, such as the selective highlighting of the question display (\emph{Sec.~3.1.2.2}), can be performed.

\subsubsection{Question Management}

Sets of questions are written as external, plain-text files contaning S-expressions corresponding to questions, separated by newlines.  This approach was chosen so that questions can be read into a list using Clojure's standard library, making it possible to navigate between them by setting the current question to the value of the desired question in the list.

A future extension to this project could be to develop a partner application to \emph{Nico} to aid tutors unfamiliar with LISP syntax to develop their own question sets in a more `user-friendly' manner.

\subsection{Interaction Handler}
% the bit that handles all the application shit: render, main-window, etc.
% talk about jfx2 vs. guiftw/swt vs. seesaw/swing
% talk about main-window here i guess?
% talk about dialogue boxes
%
% \emph{{\sc Note:} These paragraphs aren't in the correct order yet; I've just been writing things that I want to include in this subsection, and will reorder them when I have a better idea of where this bit's going.}

The interaction handler forms the `middle layer' of the application (\emph{Fig.~\ref{fig:NicoArch}}) that arbitrates between the infrastructure and the user interface.

A rendering function is called every time the mouse is moved, as every action that updates the canvas requires use of the mouse.  By only re-rendering when the mouse moves, the application does not render when the canvas is unaltered.  The function paints the canvas white, draws a bin icon in the corner (see \emph{Sec.~3.2.2.2}) and iterates across the circle list, drawing corresponding circles.  References to other circles are visualised as lines extending from the source circle to the appropriate argument slot on the recipient.

\subsubsection{GUI Libraries}

The original intention of the project was to develop \emph{Nico} using the JavaFX library. However, a number of difficulties were encountered with this: firstly, JavaFX version 2.0 was not available for the primary development environment.  Installation was attempted on a Windows machine, but caused a number of problems. It was decided to proceed using the Eclipse Foundation's SWT toolkit and Szymon Witamborski's corresponding Clojure bindings \cite{GuiFtw}.

During the development of the user interface, many problems were encountered with SWT, particularly regarding drawing arbitrary shapes on the canvas and extracting user input from dialogue boxes.  To keep the project on-schedule, the application was migrated to Java Swing, using the \emph{Seesaw} library to provide bindings for Clojure \cite{Seesaw}.  Development efficiency was increased as a result of the author's previous experience using Swing.

\subsubsection{Main Window}

At the heart of the interaction handler is the structure defining the main window, which comprises the majority of the user interface.  It controls what is displayed and responds to certain conditions.  It listens for certain events and prompts the user for input when required.  Seesaw requires interfaces to be structured such that each window is defined as one large structure, with appearance, listeners and other parameters set as options.  Other structures, such as panels and canvases, can then be used as content.

Embedded into the main window are several mouse listeners.  While the application is running, five events are listened for: motion, dragging, presses, releases and clicks.\footnote{It should be noted that a mouse click is defined by Java as a button press followed by a button release, and that, upon a click, it issues a press event, followed by a release event, followed by a click event \cite{JavaApi}.}  In addition to re-rendering, motion also triggers a check to determine if the mouse is over a circle; if so, a highlighting function is called.  This draws a ring around the circle, and performs a string comparison: using the same function that converts the questions into traditional mathematics for display in the information panel, the circle's S-expression is converted and compared to the question string.  If it is a substring, the appropriate question substring is also highlighted, to allow the user to check their work.  It was decided to implement this as a string, rather than S-expression, comparison as the string search is able to return character indices between which to highlight.

\subsubsection{Control Scheme}

Upon detection of a mouse click, the software determines the context in which the mouse was clicked, as this constitutes much of the application's control scheme.  The mouse's current co-ordinates on the canvas, as well as which mouse button was pressed and any modifier keys that were active, are used to determine the correct action to take, as illustrated in \emph{Fig.~\ref{fig:ClickTree}}.

\begin{center}
\begin{figure}[H]
\begin{center}
\includegraphics[width=\textwidth-4cm]{figs/nico_click.pdf}
\end{center}
\caption{Decision tree to determine the appropriate response to a mouse click event.}
\label{fig:ClickTree}
\end{figure}
\end{center}

A more in-depth analysis of the control scheme follows.

\begin{itemize}
\item Mouse not in circle
\begin{itemize}
\item A blank circle is created, centred on the current location of the mouse cursor.  It has two arguments, {\sfapp ?} and {\sfapp ?}, and its operator is {\sfapp ?}.
\end{itemize}
\item Mouse in circle
\begin{itemize}
\item Left-click
\begin{itemize}
\item Mouse on argument
\begin{itemize}
\item A small blue circle is drawn around the argument and a dialogue box is launched, using a spinner to set its value.
\end{itemize}
\item Mouse on operator
\begin{itemize}
\item A small blue circle is drawn around the operator and a dialogue box is launched, using four radio buttons to set its value.
\end{itemize}
\end{itemize}
\item Right-click
\begin{itemize}
\item Ctrl key pressed
\begin{itemize}
\item The final argument is removed from the circle, with a minimum of two arguments.  An alert box warns the user if they try to use fewer.
\end{itemize}
\item Ctrl key not pressed
\begin{itemize}
\item A placeholder argument is added to the circle, with a maximum of eight arguments.  An alert box warns the user if they try to use more.
\end{itemize}
\end{itemize}
\end{itemize}
\end{itemize}

Drag-and-drop functionality is also implemented.  When the application detects that a press-event has occurred, if the event was triggered whilst the cursor was over a circle, that circle is stored alongside the current system time and the button pressed in a mutable data structure.  When a drag-event is detected, the information is used to perform one of two operations upon the circle.

If the left button is held whilst dragging, the circle's co-ordinates are updated to the current co-ordinates of the cursor.  If the right button is held whilst dragging, a line is drawn from the circle to the current location of the cursor.  When the mouse passes over the argument slot of another circle, that line is fixed, and the target circle is updated such that that argument is now a reference to the source circle.

A drag-and-drop action constitutes a press-event, followed by a drag-event, followed by a release-event.  By storing information about the circle being dragged when the a press occurs, the function called upon dragging can use and modify this information easily.  This does not impact upon other events involving presses and releases (e.g. clicking), as this information is discarded upon every release, regardless of whether it was being used.

The control scheme is almost entirely mouse-based as this is a simpler interface than the keyboard, and is familiar to most computer users in a school environment (the use of mouse control, such as in many word processors, is mandated by the National Curriculum).  By limiting control to the mouse, the user is required to remember fewer buttons, and is likely to be able to guess controls if forgotten.  A caveat to this is that the use of the Ctrl key is required, as this was preferable to mouse-chording (unfamiliar to novices), or to requiring a three-button mouse (not always available).

The design of the control scheme is discussed further in \emph{Sec.~3.2.3}. % TODO: check this, gonna rearrange massively

% do i need to mention *all* the times clear-screen and render are called?  it's kind of implicit (to me anyway)...
%
% \emph{{\sc Note:} This is as far as I've got (about halfway by my reckoning).  }\verb¬texcount¬\emph{ tells me I've got about 5,800 words.}
% \subsection{User Interface}
% do we even need this section?  applciation below might cover it, and to be fair it isn't really part of the architecture - listeners and that should probs go in interaction handler above
%
\section{User Interface}%{Application}
% application in which the language is manipulated
% talk about how the app has changed, show screenshots
% talk about how the app becamse more minimalist, to focus on the language itself
% language as a main focus, with useful information n shit around it in the app
% colour-coding text (though consider colour-blindness with the red and green text)
% question highlighting - talk about troubles with it, but also merits of including it
% do we need this section?  shouldn't this be the ui bit in the prev. section?
% have lost the ui bit in favour of having this as a separate section
% target audience - children (year 5), but also applications in remedial adult learning (i.e. can't be too childish)
% hence must be appealing, clear, easy-to-read, not too verbose
% prototypes! take screenshots of old versions

The user interface comprises the parts of the application that the user directly interacts with.  The development of \emph{Nico}'s UI was key to the success of the overall project, being primarily an experiment in human-computer interaction.

During its development, \emph{Nico}'s user interface went through a number of revisions, which are detailed below.

\subsection{Application}

\begin{center}
\begin{figure}[H]
\begin{center}
\includegraphics[width=\textwidth]{figs/nico_screen_01.png}
\end{center}
\caption{A screenshot of a \emph{Nico} session.}
\label{fig:Nico1}
\end{figure}
\end{center}

The main application window has two main sections: the information panels that display the question and the current value of the calculation, as well as the {\sfapp Check} button to submit the current total as the answer; and the canvas, upon which the user constructs diagrams.  This canvas occupies most of the main window (\emph{Fig.~\ref{fig:Nico1}}), and the overall design of the application is minimal, with the aim of emphasising the diagram above all else.

\vspace{0.75cm}
\subsubsection{Development}

\paragraph{Initial Revision and Control Buttons}\hfill

\begin{center}
\begin{figure}[H]
\begin{center}
\includegraphics[width=\textwidth-4cm]{figs/nico_screen_oldest.png}
\end{center}
\caption{A very early version of the user interface.  Many features are yet to be implemented, but the button-based layout and the early revision of the notation is evident.}
\label{fig:OldApps1}
\end{figure}
\end{center}

Initially, the application was intended to have more controls immediately visible to the user (\emph{Fig.~\ref{fig:ProtoCirc}}).

The first version of the interface used buttons that appeared to the right of the canvas (\emph{Fig.~\ref{fig:OldApps1}}).  These were relocated to save screen-estate for the diagram.  Each button would cause a dialogue box to appear, which required a circle to be specified by name before any operations could be performed.  At this point, each circle was required to be labelled.

This design was abandoned as the separation between the user and their work was too great; creating circles by specifying options in dialogue boxes and watching the results appear on the canvas was not a natural way to interact with one's calculation, and hence was inappropriate for the target audience.  Forcing the user to commit to a design before seeing it entailed \emph{premature commitment}, and having to re-express circles to edit them increased \emph{viscosity}.

\paragraph{Context Menus and Automated Marking}\hfill

\begin{center}
\begin{figure}[H]
\begin{center}
\includegraphics[width=\textwidth-4cm]{figs/nico_screen_older.png}
\end{center}
\caption{A later revision of the user interface.  The screen has been cleared, allowing context menus to replace buttons.  The facility for checking an answer has been removed, due to its automation in this version.  The notation has now been fully implemented, but still lacks many features of the final version.}
\label{fig:OldApps2}
\end{figure}
\end{center}

Later revisions of the software removed the buttons and implemented a system of context menus, further increasing the space available for the user's answer, and making the consequences of actions more evident: for example, right-clicking on a circle and selecting `Delete', rather than pressing a `Delete' button and specifying the target's name.  Hence, the user was able to interact more with the canvas itself.  The dialogue boxes from the previous revision were combined into one (\emph{Fig.~\ref{fig:GUDB}}), which was used for both creation and modification.

Answer-checking was automated: when the total reached a value equal to that of the answer to the question, the user immediately progressed to the next question.  The intention here was to streamline the user experience, increasing the speed with which question sets were completed.

This design was abandoned as, again, it separated the user from their work.  Although this design was an improvement, allowing the user to interact directly with the canvas, the unified dialogue box was too complex, and still entailed \emph{premature commitment}.  The automatic answer-checking was also problematic, as it did not give the user a chance to review their answer before proceeding.  The user could not learn from their mistakes, for example if they had accidentally reached the right answer as part of a larger, incorrect calculation.

\paragraph{Final Revision}\hfill

\begin{center}
\begin{figure}[H]
\begin{center}
\includegraphics[width=\textwidth-4cm]{figs/nico_screen_new.png}
\end{center}
\caption{The final revision of the user interface.  The layout is similar to the previous version, but now includes highlighting and a bin icon for deletion.  The notation is now colour-coded, and the answer-checking button has been reintroduced.}
\label{fig:OldApps3}
\end{figure}
\end{center}

The final revision of the software reintroduced the {\sfapp Check} button to address the issues with the automatic answer-checking in the previous version.  The system of context menus was replaced with a new control scheme (\emph{Sec. 3.2.2}).

\subsubsection{Dialogue Boxes}

Dialogue boxes have been a consistent feature of the user interface throughout development.  Initially, they were triggered by buttons (\emph{Fig.~\ref{fig:OldApps1}}), and consisted of a text field, into which the user typed a circle's name, followed by, for creation and modification, an S-expression specifying the calculation that the circle was to represent.

This was never intended to be the final version, but was useful for testing the rest of the UI and interaction handler.  Requiring a user to be familiar with LISP syntax ran counter to the intentions of this project, i.e. to provide an accessible alternative notation for arithmetic for the non-programmer, pre-algebra user.

\begin{center}
\begin{figure}[H]
\begin{center}
% \includegraphics{figs/nico_gudb.png}
\includegraphics[height=\textheight/2-2cm]{figs/nico_gudb.png}
\end{center}
\caption{The unified dialogue box.}
\label{fig:GUDB}
\end{figure}
\end{center}

In later revisions of the software, the many dialogue boxes were combined into one, with many options available to specify the parameters of a circle, dynamically generating its layout to make new circles available as needed (\emph{Fig.~\ref{fig:GUDB}}).  The intention was to provide all available controls in one location.  However, this still required the user to perform operations without obvious consequences, requiring unnecessary \emph{premature commitment}.  Additionally, the act of visualising the circle from specified parameters constituted a superfluous \emph{hard mental operation}.  Finally, this dialogue box required {\bf more} time to search for information than several simpler dialogue boxes, in which less information would be presented at once.

In the final revision of the software, the unified dialogue box was reduced to two simpler ones (\emph{Fig.~\ref{fig:Dialogues}}), used in combination with the new control scheme.  The large dialogue box was unwieldy and dense, increasing the \emph{viscosity} of the system by requiring all parameters to be set every time one needed to be edited.  Two simple dialogue boxes served to rectify this, enabling single arguments or operators to be edited independently of their circles.  In combination with the improved control scheme, this helped to eliminate the \emph{premature commitment} of previous designs, by making the creation of a circle an incremental process, rather than a single event.

\begin{center}
\begin{figure}[H]
\begin{center}
\subfloat[Editing a circle's operator]{
\includegraphics[width=\textwidth/2-2cm]{figs/nico_op.png}
}
\subfloat[Editing a circle's argument]{
\includegraphics[width=\textwidth/2-2cm]{figs/nico_arg.png}
}
\end{center}
\caption{The dialogue boxes used to modify existing circles in the application.}
\label{fig:Dialogues}
\end{figure}
\end{center}

\subsection{Notation}%metaphor?

% graphical language itself
% show previous versions
% talk about how the circles have to make clear e.g. arg order
% talk about how circles had to be improved, show iterations
% more screens: old versions of the language to compare
\begin{center}
\begin{figure}[H]
\begin{center}
\subfloat[An old version of the notation.  Circles are uniformly coloured, and links feed into the centre of their target circles.  There is no indication of argument evaluation order.]{
\includegraphics[width=\textwidth/2-2cm]{figs/nico_notn_old.png}
}\hspace{1cm}
\subfloat[The final version of the notation.  Circles are colour-coded by operation, highlighting has been implemented and links fill argument slots, ending in a circle to indicate their destination.  The diagonal line and arrows have been added to circles to indicate evaluation order.]{
\includegraphics[width=\textwidth/2-2cm]{figs/nico_notn_new.png}
}
\end{center}
\caption{The notation used went through a number of revisions before reaching its current state.}
\label{fig:OldNots}
\end{figure}
\end{center}

The notation itself was revised several times.  The original vision of the notation can be seen in \emph{Fig.~\ref{fig:ProtoCirc}}.  It was decided to move away from this model: the `tails' occupied too much screen estate, and were thus replaced with simple lines, allowing many linked circles to be closer together but still \emph{visible}.

\subsubsection{Early Revisions}

The first implementation of the notation consisted of uniformly-coloured circles connected by lines feeding into the centres of their recipients, so that references would not restrict the number of available arguments per circle.  A colour scheme had not yet been implemented.

Placeholder values were unnecessary as the dialogue boxes meant that at no point could a circle have any parameter undefined.  There was also no indication of argument ordering, though it was defined.  This was acceptable for commutative operations, but became confusing when this was not the case. The user was required to determine the order through experience, running counter to the intentions of this project: a significant period of initiation should not be required, especially due to notational deficiencies.

Another, greater, problem related to argument ordering was that of links between circles.  With all links pointing to the centre of their targets, it was not at all clear in which order they were evaluated, neither relative to each other nor to the other arguments.  This was problematic in situations where one circle was used as an argument to several others.

% TODO: poss. a figure showing close-ups/details of newest notation?

\subsubsection{Final Revision}
The design of the notation was changed to address these problems.  Firstly, the circles are now colour-coded by operator, making it easier to distinguish between them.  The circles themselves have also been embellished with the addition of a line indicating the first argument.  Arrows have also been added to the edge of the circles to indicate in which direction the arguments are evaluated.  Finally, the links have been changed to occupy an argument slot, with a marker on the end of the link line to clearly show where it terminates.
% TODO: more cogdims!  ALWAYS more cogdims!
% TODO: testing?  or does that go in the next chapter?

\subsection{Control Scheme}

% TODO: diagram?  like the one in the notebook, showing the consequences of each possible action to illustrate the control scheme and notation
% why the '(\? \? \?) initial circle expr?

\subsubsection{Development}

The original collection of buttons and dialogue boxes to control the circles was implemented to ensure that the user interface was functional, and so that circle operations could be tested.  It was abandoned due to its required \emph{premature commitment}.

In later revisions, a system of context menus was implemented to decrease the separation between the user and their work.  Right-clicking on a blank patch of the canvas gave the user the option to create a new circle, and right-clicking on an existing circle presented the user with the option to either edit or delete it.  This required less forethought on the part of the user, but the use of the unified dialogue box (\emph{Fig.~\ref{fig:GUDB}}) still entailed \emph{premature commitment}.

The control system is, in its final revision, predominantly mouse-based for the construction of calculations in the application's notation.

\subsubsection{Final Revision}

Left-clicking on a blank patch of canvas creates a circle containing a placeholder expression.  This approach was chosen to make it clear to the user that the new circle required attention, and so that it wouldn't interfere with evaluation, (see \emph{Sec. 3.1.2.3}).

Left-clicking on a circle's operator brings up a dialogue box with a radio button for each operator (\emph{Fig.~\ref{fig:Dialogues}}).  Similarly, left-clicking on a circle's argument brings up a dialogue box containing a spinner to set a new value, between -10 and 10.  This limitation was introduced to prevent the application from being used as a calculator.  The limited range of numbers means that the user must still consider strategies such as partitioning or long multiplication, focussing on how to solve the problem, rather than simply transcribing the question into one circle that will complete the calculation for them.  The user is not required to calculate manually using numbers within this range, as it is assumed that the target audience is familiar with very simple arithmetic.

Right-clicking on a circle increments its number of arguments by one, to a maximum of eight.  Holding Ctrl and right-clicking on a circle decrements its number of arguments by one, to a minimum of two.  An alert box appears if these boundaries are exceeded.  The limit on arguments was decided upon as fewer than two arguments is an unfamiliar concept that is rarely of use to the target audience.  Although Clojure is able to evaluate, for example, \verb¬(+ 1)¬ (returning \verb¬1¬), this could confuse the user.  The upper limit of eight arguments prevents circles from becoming overcrowded and illegible.

Circles can be repositioned by left-clicking and dragging.  If a circle is dragged to the bin icon, it is removed.  Originally, deletion was mapped to a mouse control, but to simplify the control scheme, it was decided that a familiar icon should be used instead.  This also has the advantage of implying that mistakes can be erased, encouraging exploration.

To link circles, right-clicking and dragging from a circle allows a line to be drawn to a slot on the recipient.  By having the user pull links out from existing circles, as opposed to specifying references and links consequently appearing, the direction of dataflow is emphasised.

This new control scheme allows the user to directly manipulate their work in progress, with each action having clear consequences.  It reduces the \emph{viscosity} of editing circles from the previous control schemes, and reduces the \emph{premature commitment} required when creating circles.  The environment encourages exploration, and reduces the time spent searching for information within the interface.

\subsection{Colour-Coding}

% talk about colour-blindness
% why this colour scheme? psychological reasons?
% why not random or connection-based colours?
\emph{Nico} uses colour-coding in various areas of the application, to draw attention to important elements.  Colour-coding has been shown to faciliate more effective learning.  Lamberski and Dwyer conclude that, for younger learners, colour facilitates performance due to its ``attention or motivation elements''.  They state that
\begin{center}
\parbox[c]{\textwidth-2cm}{
\small
``when younger learners have been given self-paced instructional materials containing color-coded reading and phonic concepts, similar positive results have been attained.'' \cite{Lamberski1983}
}
\end{center}
Colour was originally intended to signify how nested a circle was (\emph{Fig.~\ref{fig:ProtoCirc}}).  This was rejected as it could become confusing if circles were reused (\emph{Fig.~\ref{fig:Nico2}}).  Consequently, circles are coloured according to their operators, clearly separating different operations and increasing \emph{visibility of notation}.

\begin{center}
\begin{figure}[H]
\begin{center}
\includegraphics[width=\textwidth-2cm]{figs/nico_screen_02.png}
\end{center}
\caption{An example of where nesting-based colour-coding would be confusing.  The circle containing (2+2) can be considered to be both one and two circles removed from the root circle (the colour scheme used here is determined by operator).}
\label{fig:Nico2}
\end{figure}
\end{center}

Colour-coding is used in the information panels: positioning the cursor over a circle highlights the part of the question that it is likely to represent in blue, and highlights the circle with a blue outline (see \emph{Fig.~\ref{fig:Highlighting}}).  This helps the user to assess their progress towards a solution, and allows them to focus their attention upon a single circle.

When an answer is submitted, the text in the answer panel turns red if the answer was incorrect, or green if the answer was correct, which provides an unintrusive form of feedback, allowing the user to continue uninterrupted if their answer is wrong.

\begin{center}
\begin{figure}[H]
\begin{center}
\subfloat[]{\includegraphics[height=\textwidth/3-1.5cm]{figs/nico_hi1.png}}
\subfloat[]{\includegraphics[height=\textwidth/3-1.5cm]{figs/nico_hi2.png}}
\\
\vspace{0.5cm}
\subfloat[]{\includegraphics[height=\textwidth/3-1.5cm]{figs/nico_hi3.png}}
\end{center}
\caption{Three instances of the question text being highlighted as a corresponding circle is moused-over.}
\label{fig:Highlighting}
\end{figure}
\end{center}

\begin{center}
\begin{figure}[H]
\begin{center}
\subfloat[Red text indicates that the answer submitted was incorrect]{
\includegraphics[width=\textwidth/3-1cm]{figs/nico_wrong.png}
}\hspace{2cm}
\subfloat[Green text indicates a correct answer; a dialogue box allows the user to progress to the next question when they are ready]{
\includegraphics[width=\textwidth/3-1cm]{figs/nico_right.png}
}\\
\subfloat[The circle's operator is highlighted when it is being edited]{
\includegraphics[width=\textwidth/3-1cm]{figs/nico_op_hi.png}
}\hspace{2cm}
\subfloat[An argument is highlighted when it is being edited]{
\includegraphics[width=\textwidth/3-1cm]{figs/nico_arg_hi.png}
}
\end{center}
\caption{Use of colour in the \emph{Nico} user interface.}
\label{fig:ColourCoding}
\end{figure}
\end{center}

\subsubsection{Colour Blindness}

As \emph{Nico}'s notation and user interface are reliant upon colour to express or emphasise information,  some parts of the application may be inaccessible to colour-blind users.  Although no functionality to address this was included in the application itself, the colours are easily replaced in the source code.  They are defined early on as variables: by changing their definitions, it is possible to create customised versions of the software to suit individual cases of colour blindness.  For example, a person with total colour blindness may use shades of grey, rather than colours, whereas people with other forms of colour blindness, such as protanopia or tritanomaly, may only need to replace certain colours.% TODO: implement this! %TODONE!

\section{Testing}

Development of the software was primarily REPL-driven, allowing for rapid development and testing whilst the application was running.  A suite of unit tests were derived from assertions made in the REPL, using Leiningen's testing framework and the \verb¬clojure.test¬ library.  These constituted a number of tests that were able to be run all at once by issuing a command.  This was advantageous as it allowed for quick assessment of many functions.

Unit tests were performed on many functions, especially those that did not require access to mutable state.  Such functions behaved predictably.  As pure functions with no side effects, called upon immutable data, these did not require further testing once unit-tested.  An example of a unit test is shown in \emph{Fig.~\ref{fig:LeinTest}}.

\begin{center}
\begin{figure}[H]
\begin{center}
\begin{minipage}{10.9cm}
\begin{Verbatim}[fontsize=\small,numbers=left]
(deftest lisp-maths
  (is (= (lisp-to-maths '(+ 2 3)) "2+3"))
  (is (= (lisp-to-maths '(- 2 3)) "2-3"))
  (is (= (lisp-to-maths '(* 2 3)) "2×3"))
  (is (= (lisp-to-maths '(/ 2 3)) "2÷3"))
  (is (= (lisp-to-maths '(+ 19 28 30 1 -4)) "19+28+30+1+-4"))
  (is (= (lisp-to-maths '(+ (- 2 3) (* 4 5))) "(2-3)+(4×5)"))
  (is (= (lisp-to-maths '(+ 3 4 (- 2 1) 1)) "3+4+(2-1)+1")))
\end{Verbatim}
\end{minipage}
\end{center}
\caption{A unit test taken from the test file.  Several assertions are made within the body of a test.  The command {\ttfamily lein test} causes all such tests to be run, providing the developer with the number of passes and failures, and details of any failures that occurred.}
\label{fig:LeinTest}
\end{figure}
\end{center}

Integration testing was required for larger components, particularly those that accessed mutable data structures.  The REPL-driven nature of development meant that it was possible to evaluate certain statements while the application was running, for example to compare the updated value of a circle after a modification operation (\emph{Fig.~\ref{fig:ReplTest}}).

\begin{center}
\begin{figure}[H]
\begin{center}
\begin{minipage}{9.5cm}
\begin{Verbatim}[fontsize=\small,numbers=left]
nico.core> (:circ (find-circle "c0"))
(#<core$_PLUS_ clojure.core$_PLUS_@24de1f47> 1 2 3)
nico.core> (remove-last-arg (find-circle "c0"))
[redacted: output stating successful update of agent]
nico.core> (:circ (find-circle "c0"))
(#<core$_PLUS_ clojure.core$_PLUS_@24de1f47> 1 2)
\end{Verbatim}
\end{minipage}
\end{center}
\caption{A REPL-based integration test for a function to remove the final argument of a circle.  The expression contained within a circle is checked, the removal function is called, and the expression is checked again.  The effect is visible; the final argument has been removed.}
\label{fig:ReplTest}
\end{figure}
\end{center}

Testing was made much easier by the small number of mutable data structures.  Fewer integration tests were required, as after it was established that a pure function that did not interact with mutable objects behaved as expected, it could be assumed that it would behave as expected outside of the unit-testing situation.

%5Throughout the development of the software, unit tests were performed to ensure the integrity of the code.  GNU Emacs, in combination with SLIME/SWANK and clojure-mode, includes functionality such that commented-out lines of code can be evaluated at will.  As such, it was possible to write unit and integration tests into the main body of my code and to evaluate them whenever a change may have affected how a function behaved.  In the interests of tidiness and code readability, many tests were removed as they became unnecessary later on in the development in the project.
%
%% TODO: chat shit about assertions
%Assertions were also used to perform tests: Clojure provides the developer with the ability to specify runtime tests as part of function definition, throwing exceptions if any assertions.  Such tests were used in addition to the previous method to unit-test and integration-test the functions within the application.
%
\section{Summary}

The software fulfils and exceeds the criteria for success outlined in the project proposal, being a usable system that improves upon handwritten arithmetic.

Third-party libraries, in particular Seesaw, were used where necessary.

The application comprises three sections: the infrastructure, handling mathematical operations and memory management; the interaction handler, acting as an interface between the user interface and the infrastructure; and the UI itself.

Mutable state was introduced where necessary, but kept to a minimum.  The application uses just six mutable data structures, meaning that functions not interacting with these objects will behave predictably, making the software stabler, and easier to comprehend and test.

The UI was revised many times to correct deficiencies in the design that would have led to a poor user experience, and the infrastructure and interaction handler were also both restructured as new demands upon the software were made.

\cleardoublepage
\chapter{Evaluation}

This chapter details the results of tests carried out to evaluate the software.  First is an analysis of the time taken to update the display, followed by the results of the user study.

%\section{Testing}% {Defect Management and Testing}% {Backend Testing}
% used slime/swank with test lines as comments to be evaluated with C-x C-e
% emacs/slime/swank as ide; ideal for lisps!
% lein test framework - need to actually set up some tests first though!
% clojure test-is; see http://en.wikibooks.org/wiki/Clojure_Programming/Tutorials_and_Tips#Unit_Testing_in_Clojure
%
% does C-x C-e *actually* count as unit testing?  i mean, i *did* test small bits and then test bigger bits after i'd established that the smaller bits worked...
% apparently it does! w00t
% also talk about integration testing (after the units have been tested, testing the bits that use those units), and see the link in alistair's email re other levels of testing
%Throughout the development of the software, I performed unit tests to ensure
%the integrity of my code.  My development environment of choice, GNU Emacs
%using SLIME/SWANK and clojure-mode, includes functionality such that
%commented-out lines of code can be evaluated at will.  As such, it was possible
%for me to write unit and integration tests into the main body of my code, evaluating them
%whenever a change may have affected how a function behaved.  In the interests
%of tidiness and code readability, many tests were removed as they became
%unnecessary later in the development in the project, but \emph{Fig.~\ref{fig:SlimeTests}}
%shows one of the test blocks that were left in the most recent revision of the
%source code.
%
\section{UI Evaluation}
% user testing
% graphs
% 300ms test - see below (under goals)
% maybe time render too?

The time taken to generate an S-expression, evaluate it and display the result to the user, implemented in the function \verb¬update-answer¬, was required to be less than 300ms.  To test this, a \emph{Nico} session was completed, using the same questions as the user study, but with a modified version of \verb¬update-answer¬ that logged the system time upon entering and exiting the function in an external file.

A plot of the time taken for each call to \verb¬update-answer¬ to complete over time is shown below.

\newpage
\begin{landscape}
\begin{center}
\begin{figure}[H]
\begin{center}
\subfloat[]{
\includegraphics[height=\textheight-4cm]{figs/graphs/update_times_full.png}
}
\subfloat[]{
\includegraphics[height=\textheight-4cm]{figs/graphs/update_times_cut30.png}
}
\end{center}
\caption{\emph{a)} plot and \emph{b)} detail of the time taken for {\ttfamily update-answer} to complete over the course of one session.}
\label{fig:UpdateTimes}
\end{figure}
\end{center}
\end{landscape}

\newpage
\verb¬update-answer¬ was, in almost all cases, able to complete in significantly less than the limit of 300ms, with a mean time of 6.751852ms.  Of the 810 samples taken, there was one anomalous time of 308ms.

% \subsection{Notation Evaluation}
% % user testing
% % graphs
%
\section{User Study}
% results
% pilot study - ellie suggested switching left/right-clicks in control scheme, implemented before main study

To assess the utility of the software relative to handwritten arithmetic, a user study was conducted to evaluate the system in a more realistic context.  Users completed tasks both by hand and using the software, and completed a questionnaire (\emph{Appendix C}).

\subsection{Experimental Design}

The original intention was to perform a user study upon a Year 5 `test class', but working with vulnerable individuals involved ethical complications that were diffcult to overcome, given the time constraints of this dissertation.  In addition to obtaining the approval of the Ethics Committee and the permission of a school to conduct the study, consent forms would have to be distributed to the test class to be signed by the pupils' guardians and returned via the school to the author.  It was therefore decided that students not currently reading a `mathematical' subject such as the Computer Science, Natural Sciences or Mathematical Triposes, were suitable alternatives.

\subsubsection{Pilot Study}
% changes, improvements

A pilot study was conducted with one participant, to assess the feasibility of the experiment.  The outcome was to make changes to several areas of the test, and a small change to the application itself.

Firstly, the participant noted that the original version of the tutorial video was hard to follow; it used subtitles to convey information, which were felt to be hard to read and understand in their time on-screen.  The participant suggested using a voiceover instead.  They also noted that some of the questions in the feedback section were confusing or irrelevant, and was unsure of how to answer them.  Finally, the participant felt that the control scheme of the application seemed counterintuitively focussed upon the right mouse-button, where it would have felt more natural to have used the left.

As a result, the video was reproduced using a voiceover rather than subtitles, and irrelevant questions were removed.  Finally, the control scheme was revised, assigning the left mouse-button to primary functions such as the creation and editing of circles, and the right to functions such as changing the number of arguments.

\subsubsection{Main Study}

Participants were first asked to sign a statement of informed consent to participate in the study, and were asked if they had previously used \emph{Nico}. They were required to indicate their mathematical confidence using a five-point Likert item.  The meaning of each point in the Likert item was explained to the users beforehand, as it is well-established that people tend to overestimate their abilities \cite{Mura1987}.

Users were divided into two groups.  The first watched a tutorial video and explored the application in a `sandbox' environment for five minutes.  They then completed one question set using \emph{Nico}, and then answered the same questions by hand.  The second group completed the same questions by hand, before watching the video and using the software in the same way as the first group.  This reversal was to eliminate any advantage exhibited by having completed the questions once, before doing so again using a different method.

The time taken to answer each question was recorded.  A means of logging the system time at significant points in the user's progress was implemented in \emph{Nico}, and the handwritten questions were timed using a stopwatch.

Both groups were then asked to complete a series of questions to provide feedback on the software, using questions from Blackwell and Green \cite{Blackwell2000}.

% TODO: You need an experimental design section that describes why you used the cognitive dimensions questionnaire for evaluation instead of other means of evaluation, why you compared paper questions with nico instead of a calculator, why you used a tutorial video instead of a text based tutorial, why you used a likert scale instead of another scale, why you used time as a metric of comparison,
This approach was chosen for a number of reasons.  \emph{Nico} was compared to questions answered by hand, rather than a calculator, as its intention is not to calculate answers for the user, but to be used to visualise method.  To compare \emph{Nico} to calculator use would not be meaningful, as the use of a calculator entails different methods to those that would be used when calculating by hand.

The decision to use two methods of evaluation of the software (timing users and the `Cognitive Dimensions' questionnaire) allows several aspects of the software to be assessed.  By timing responses to questions both by hand and using the software, the time taken by participants to think about how to answer a question is gauged.  It is expected that \emph{Nico} should decrease the amount of time taken to work through complex problems involving several subcalculations.

The questionnaire also allows users to critique the design of the user interface; this data was used to assess its reception.  With regard to the design of the questionnaire itself, a Likert item was used to assess the users' initial level of confidence as this provides a quantifiable result, and thus is analysed more easily than a verbal response.

A tutorial video was decided upon, rather than a textual approach or a live demonstration, as it provides a means of demonstrating an example session in a standardised way.  In a live demonstration, important elements could potentially be omitted.  A textual description may not make certain processes clear, and may use unfamiliar terminology.

\subsection{Results}
% data, graphs, analyses
% p-values:
% q1:  t = -8.0204, df = 7, p-value = 8.968e-05 <- sig
% q2:  t = -8.0493, df = 7, p-value = 8.764e-05 <- sig
% q3:  t = -8.9296, df = 7, p-value = 4.489e-05 <- sig
% q4:  t = -2.9592, df = 7, p-value = 0.02113   <- sig
% q5:  t = -5.288, df = 7, p-value = 0.001138   <- sig
% q6:  t = -1.3149, df = 7, p-value = 0.2300
% q7:  t = 2.3688, df = 7, p-value = 0.0497     <- sig
% q8:  t = 3.1, df = 7, p-value = 0.01732       <- sig
% q9:  t = 4.0293, df = 7, p-value = 0.005      <- sig
% q10: t = 2.1298, df = 7, p-value = 0.0707

The results of the tests are shown in \emph{Figs.~\ref{fig:StacksGp1}} and \emph{\ref{fig:StacksGp2}}, illustrating the times taken by the participants in each group respectively to answer the questions both by hand and using \emph{Nico}.

%\begin{center}
%\begin{table}[H]
%\begin{center}
%\begin{tabular}{|r||r|r||r|r||r|r||r|r|}
%\cline{2-9}
%\multicolumn{1}{c|}{} & \multicolumn{8}{c|}{Group 1}\\ \cline{2-9}
%\multicolumn{1}{c|}{} & \multicolumn{2}{c||}{Participant 1} & \multicolumn{2}{c||}{Participant 2} & \multicolumn{2}{c||}{Participant 3} & \multicolumn{2}{c|}{Participant 4}\\ \hline
%\multicolumn{1}{|c||}{Question} & \multicolumn{1}{c|}{Hand} & \multicolumn{1}{c||}{\emph{Nico}} & \multicolumn{1}{c|}{Hand} & \multicolumn{1}{c||}{\emph{Nico}} & \multicolumn{1}{c|}{Hand} & \multicolumn{1}{c||}{\emph{Nico}} & \multicolumn{1}{c|}{Hand} & \multicolumn{1}{c|}{\emph{Nico}}\\ \hline \hline
%Q1 & 4600 & 19547 & 3100 & 34094 & 1800 & 24657 & 1300 & 38891\\ \hline
%Q2 & 8800 & 23313 & 7000 & 24329 & 6300 & 17437 & 5400 & 12031\\ \hline
%Q3 & 20600 & 53500 & 13600 & 36360 & 10100 & 40798 & 10700 & 31813\\ \hline
%Q4 & 36200 & 54375 & 26400 & 60156 & 14200 & 79704 & 23800 & 50094\\ \hline
%Q5 & 61900 & 103344 & 39600 & 79844 & 26500 & 103221 & 33400 & 61064\\ \hline
%Q6 & 213400 & 224406 & 190300 & 215376 & 84900 & 172940 & 181700 & 154691\\ \hline
%Q7 & 235200 & 165813 & 1996000 & 103376 & 909000 & 48922 & 194200 & 52595\\ \hline
%Q8 & 259500 & 306625 & 282500 & 158094 & 147200 & 138581 & 256700 & 155347\\ \hline
%Q9 & 304400 & 230953 & 313500 & 123626 & 159800 & 183722 & 271700 & 196144\\ \hline
%Q10 & 416500 & 307734 & 357700 & 213642 & 197100 & 245161 & 322200 & 235364\\ \hline
%\multicolumn{9}{c}{}\\ \cline{2-9}
%\multicolumn{1}{c|}{} & \multicolumn{8}{c|}{Group 2}\\ \cline{2-9}
%\multicolumn{1}{c|}{} & \multicolumn{2}{c||}{Participant 1} & \multicolumn{2}{c||}{Participant 2} & \multicolumn{2}{c||}{Participant 3} & \multicolumn{2}{c|}{Participant 4}\\ \hline
%\multicolumn{1}{|c||}{Question} & \multicolumn{1}{c|}{Hand} & \multicolumn{1}{c||}{\emph{Nico}} & \multicolumn{1}{c|}{Hand} & \multicolumn{1}{c||}{\emph{Nico}} & \multicolumn{1}{c|}{Hand} & \multicolumn{1}{c||}{\emph{Nico}} & \multicolumn{1}{c|}{Hand} & \multicolumn{1}{c|}{\emph{Nico}}\\ \hline \hline
%Q1 & 1200 & 22299 & 2100 & 23516 & 2600 & 16142 & 2600 & 21985\\ \hline
%Q2 & 2200 & 24530 & 5600 & 19015 & 7200 & 16875 & 7200 & 23422\\ \hline
%Q3 & 17300 & 47620 & 7600 & 29735 & 18400 & 27923 & 15900 & 48470\\ \hline
%Q4 & 23000 & 50704 & 20700 & 43094 & 25300 & 37516 & 38600 & 174160\\ \hline
%Q5 & 46600 & 134828 & 33300 & 70313 & 35500 & 58954 & 72400 & 98846\\ \hline
%Q6 & 199900 & 629835 & 136400 & 164891 & 99000 & 113659 & 266600 & 252708\\ \hline
%Q7 & 202400 & 80277 & 142200 & 48890 & 1106000 & 52954 & 272900 & 72767\\ \hline
%Q8 & 295900 & 248736 & 182000 & 107375 & 184700 & 78001 & 348900 & 257224\\ \hline
%Q9 & 312900 & 152185 & 204100 & 118594 & 200500 & 67485 & 384500 & 144753\\ \hline
%Q10 & 357300 & 238300 & 261600 & 203110 & 237500 & 324581 & 472900 & 270177\\
%\hline
%\end{tabular}
%\end{center}
%\caption{Table showing the time in milliseconds taken by each participant to complete each question, by hand and using the software.}
%\label{tab:BigTimes}
%\end{table}
%\end{center}
%
\begin{center}
\begin{figure}[H]
\begin{center}
\includegraphics[height=\textheight/2-2cm]{figs/graphs/stacked-bar-times-gp1.png}
\end{center}
\caption{Time taken in milliseconds for each participant in Group 1 to complete the questions, both by hand and using the software.  Names are formatted as P\emph{n}[HC], where P\emph{n} is Participant \emph{n}, H denotes Handwritten and C denotes Computer.}
\label{fig:StacksGp1}
\end{figure}
\end{center}

\begin{center}
\begin{figure}[H]
\begin{center}
\includegraphics[height=\textheight/2-2cm]{figs/graphs/stacked-bar-times-gp2.png}
\end{center}
\caption{Time taken in milliseconds for each participant in Group 2 to complete the questions, both by hand and using the software.  Names are formatted as in \emph{Fig.~\ref{fig:StacksGp1}}.}
\label{fig:StacksGp2}
\end{figure}
\end{center}

This data is visualised using a series of plots, below.  The complete set of plots and full table of times can be found in \emph{Appendix B}.

\begin{center}
\begin{figure}[H]
\begin{center}
\includegraphics[width=\textwidth-5cm]{figs/graphs/q3.png}
\end{center}
\caption{Time taken in milliseconds per participant to answer Question 3: 1+2+3+4+5.}
\label{fig:PlotQ3}
\end{figure}
\end{center}

\begin{center}
\begin{figure}[H]
\begin{center}
\includegraphics[width=\textwidth-5cm]{figs/graphs/q6.png}
\end{center}
\caption{Time taken in milliseconds per participant to answer Question 6: ((1+2+3+4+5)+(2×4×6×8×10))×1×2×(1+2+3+4+5).}
\label{fig:PlotQ6}
\end{figure}
\end{center}

\begin{center}
\begin{figure}[H]
\begin{center}
\includegraphics[width=\textwidth-5cm]{figs/graphs/q7.png}
\end{center}
\caption{Time taken in milliseconds per participant to answer Question 7: 12+14.}
\label{fig:PlotQ7}
\end{figure}
\end{center}

\begin{center}
\begin{figure}[H]
\begin{center}
\includegraphics[width=\textwidth-5cm]{figs/graphs/q9.png}
\end{center}
\caption{Time taken in milliseconds per participant to answer Question 9: 120÷((2×10)+5+5).}
\label{fig:PlotQ9}
\end{figure}
\end{center}

\subsubsection{Statistical Analyses}

% remove df (degrees of freedom)?
\begin{center}
\begin{table}[H]
\begin{center}
\begin{tabular}{|r|r|r|r|}
\hline
Question & \emph{t} & \emph{p} & Significant?\\ \hline \hline
Q1 & -8.0204 & 0.00008968 & Yes\\ \hline
Q2 & -8.0493 & 0.00008764 & Yes\\ \hline
Q3 & -8.9296 & 0.00004489 & Yes\\ \hline
Q4 & -2.9592 & 0.02113 & Yes\\ \hline
Q5 & -5.288 & 0.001138 & Yes\\ \hline
Q6 & -1.3149 & 0.2300 & No\\ \hline
Q7 & 2.3688 & 0.0497 & Yes\\ \hline
Q8 & 3.1 & 0.01732 & Yes\\ \hline
Q9 & 4.0293 & 0.005 & Yes\\ \hline
Q10 & 2.1298 & 0.0707 & No\\
\hline
\end{tabular}
\end{center}
\caption{Table showing the results of the paired \emph{t}-test comparing the distribution of times taken to answer each question by hand and using the software, using an \emph{α} value of 0.05.}
\label{tab:TTests}
\end{table}
\end{center}

A two-sample, paired \emph{t}-test using an \emph{α} value of 0.05 was used to determine the significance of the difference between the mean time taken to answer each question by hand and using \emph{Nico}.  The paired \emph{t}-test was chosen as the samples are not independent.  The results of the \emph{t}-tests on each question's datasets are shown in \emph{Table \ref{tab:TTests}}.  The data are assumed to fit a normal distribution: a Shapiro-Wilk test was considered as a test for normal distributions, but the sample size of eight data points per set was deemed to be too small for such a test to be meaningful.

As the \emph{p}-values in the above table indicate, there was a significant difference in performance between using \emph{Nico} and completing the questions by hand.

For the first five questions, the completion time was markedly increased by using the software.  This is somewhat to be expected;  these questions were included to allow the user to get used to solving simple problems using \emph{Nico}.  They each required a maximum of three separate calculations, and constructed from numbers less than ten.

 Question 6 still used numbers less than ten, but involved several nested calculations.  There is no longer a statistically-significant difference between using the software and completing the questions by hand, suggesting that \emph{Nico} offered some improvement over handwriting that was offset by the greater time required to input simple expressions into the application.

The final four questions introduced numbers greater than ten, and included many subcalculations.  For the seventh, eighth and ninth questions, \emph{Nico} offered a statistically-significant improvement.  These questions required strategies such as partitioning or long multiplication that go beyond simply recalling number bonds.  It is these types of questions that \emph{Nico} is intended to help with, and the results demonstrate an obvious decrease in the time taken to solve them.

The tenth question, on reflection, was not well-chosen.  It is comparable in difficulty to the fifth question, constituting a series of simple subcalculations using numbers less than ten.  Though there are more calculations to perform,  this question was not challenging to solve by hand, and as such the lack of difference between the two conditions was unsurprising.  The simple subcalculations are easier to perform by hand, but combining them is faster using \emph{Nico}.

This conclusion is borne out by the plots shown above: \emph{Fig.~\ref{fig:PlotQ3}} shows a clear advantage to performing a simple calculation by hand.  Question 6 (\emph{Fig.~\ref{fig:PlotQ6}}), implies that there is no clear advantage to using either method for this type of question.  Questions 7 and 9 (\emph{Fig.~\ref{fig:PlotQ7}} and \emph{Fig.~\ref{fig:PlotQ9}}), which are `complex' questions, show \emph{Nico} to be a faster means of calculation in such cases.  This is especially the case with Question 7.  Data were also collected regarding how well the questions were answered by each user, which follow.

The Pearson product-moment correlation coefficient \emph{r} of this data was calculated, comparing the concatenated `marks' and `tries' data series (\emph{Table \ref{tab:MarksTries}}), giving two series of 80 elements.  The \emph{r} value obtained was -0.2278926.  Using Fisher's critical value of 0.2172, we see that this is significant \cite{Fisher1990}.  This indicates that \emph{Nico} does not have a negative effect on a user's ability to answer questions correctly.
%To determine the significance of this result, the \emph{r} value is compared to a critical value taken from Fisher's table, which gives, for a test at the 0.05 level and for 80 samples, a critical value of 0.2172 \cite{Fisher1990}.  Hence, the calculated \emph{r} value shows a statistically-significant, negative correlation between the marks obtained for a handwritten answer and the number of wrong submissions using \emph{Nico}.

\begin{center}
\begin{table}[H]
\begin{center}
\begin{tabular}{|r||r|r||r|r||r|r||r|r|}
\cline{2-9}
\multicolumn{1}{c|}{} & \multicolumn{8}{c|}{Group 1}\\ \cline{2-9}
\multicolumn{1}{c|}{} & \multicolumn{2}{c||}{Participant 1} & \multicolumn{2}{c||}{Participant 2} & \multicolumn{2}{c||}{Participant 3} & \multicolumn{2}{c|}{Participant 4}\\ \hline
\multicolumn{1}{|c||}{Question} & \multicolumn{1}{c|}{Marks} & \multicolumn{1}{c||}{Errors} & \multicolumn{1}{c|}{Marks} & \multicolumn{1}{c||}{Errors} & \multicolumn{1}{c|}{Marks} & \multicolumn{1}{c||}{Errors} & \multicolumn{1}{c|}{Marks} & \multicolumn{1}{c|}{Errors}\\ \hline \hline
Q1 & 1 & 0 & 1 & 0 & 1 & 0 & 1 & 0\\ \hline
Q2 & 1 & 0 & 1 & 0 & 1 & 0 & 1 & 0\\ \hline
Q3 & 1 & 0 & 1 & 0 & 1 & 0 & 0 & 0\\ \hline
Q4 & 1 & 0 & 1 & 0 & 1 & 0 & 1 & 0\\ \hline
Q5 & 1 & 0 & 1 & 0 & 1 & 0 & 1 & 0\\ \hline
Q6 & 0 & 0 & 0 & 0 & 0 & 0 & 0 & 0\\ \hline
Q7 & 1 & 0 & 1 & 0 & 1 & 0 & 1 & 0\\ \hline
Q8 & 0 & 0 & 1 & 1 & 1 & 0 & 1 & 0\\ \hline
Q9 & 0 & 0 & 1 & 0 & 1 & 1 & 1 & 1\\ \hline
Q10 & 1 & 0 & 0 & 0 & 1 & 0 & 1 & 0\\ \hline
\multicolumn{9}{c}{}\\ \cline{2-9}
\multicolumn{1}{c|}{} & \multicolumn{8}{c|}{Group 2}\\ \cline{2-9}
\multicolumn{1}{c|}{} & \multicolumn{2}{c||}{Participant 1} & \multicolumn{2}{c||}{Participant 2} & \multicolumn{2}{c||}{Participant 3} & \multicolumn{2}{c|}{Participant 4}\\ \hline
\multicolumn{1}{|c||}{Question} & \multicolumn{1}{c|}{Marks} & \multicolumn{1}{c||}{Errors} & \multicolumn{1}{c|}{Marks} & \multicolumn{1}{c||}{Errors} & \multicolumn{1}{c|}{Marks} & \multicolumn{1}{c||}{Errors} & \multicolumn{1}{c|}{Marks} & \multicolumn{1}{c|}{Errors}\\ \hline \hline
Q1 & 1 & 0 & 1 & 0 & 1 & 0 & 1 & 0\\ \hline
Q2 & 1 & 0 & 1 & 0 & 1 & 0 & 1 & 0\\ \hline
Q3 & 1 & 0 & 1 & 0 & 1 & 0 & 1 & 0\\ \hline
Q4 & 1 & 0 & 1 & 0 & 1 & 0 & 1 & 2\\ \hline
Q5 & 0 & 1 & 1 & 0 & 1 & 0 & 1 & 0\\ \hline
Q6 & 0 & 1 & 1 & 0 & 0 & 0 & 0 & 0\\ \hline
Q7 & 1 & 0 & 1 & 0 & 1 & 0 & 1 & 0\\ \hline
Q8 & 0 & 0 & 1 & 0 & 1 & 0 & 1 & 1\\ \hline
Q9 & 1 & 0 & 1 & 0 & 1 & 0 & 1 & 0\\ \hline
Q10 & 0 & 0 & 1 & 0 & 0 & 5 & 1 & 0\\
\hline
\end{tabular}
\end{center}
\caption{Table comparing the mark obtained for each question answered by hand (a correct answer scored 1, whereas an incorrect answer scored 0) to the number of attempts made to answer each question before submitting the correct answer in \emph{Nico}.}
\label{tab:MarksTries}
\end{table}
\end{center}

% pearson product-moment correlation coefficient (pearson's r)?
% R: cor.test(x,y,alternative="g",method="p") to test if x, y have pos. corr. ("l" instead of "g" for neg.)
% should maybe concat all marks columns and all wrongs columns, then test marks against wrongs for marks wrongs
% or perhaps concat as above, then separate into the 1-markers and 0-markers
% then again, might be better to just take a mean of the wrongs with that data
% could even do this by question, e.g. if they got q1 wrong on paper they're more likely to get it wrong in the software, but getting q2 wrong might not mean shit

\subsubsection{Feedback Analysis}
% feedback innit
% TODO: this section

The final section of the questionnaire constituted a series of questions adapted from Blackwell and Green \cite{Blackwell2000}.  The users' responses were graded based upon their tone, scoring -1 for negative, 0 for neutral and 1 for positive.  The total number of positive, negative and neutral response are shown in \emph{Table \ref{tab:FeedbackScores}}.

\begin{center}
\begin{table}[H]
\begin{center}
\begin{tabular}{|r||r|r|r|}
%\cline{2-4}
\hline
\multicolumn{1}{|c|}{Question} & \multicolumn{1}{c|}{Positive} & \multicolumn{1}{c|}{Neutral} & \multicolumn{1}{c|}{Negative}\\ \hline \hline
2.1.1 & 6 & 1 & 1\\ \hline
2.1.2 & 2 & 0 & 6\\ \hline
2.1.3 & 8 & 0 & 0\\ \hline
2.2.1 & 7 & 1 & 0\\ \hline
2.2.2 & 4 & 0 & 4\\ \hline
2.3.1 & 1 & 2 & 5\\ \hline
2.3.2 & 2 & 3 & 3\\ \hline
2.4.1 & 6 & 0 & 2\\ \hline
2.4.2 & 3 & 0 & 5\\ \hline
2.5.1 & 6 & 0 & 2\\ \hline
2.5.2 & 4 & 3 & 1\\ \hline
2.5.3 & 6 & 2 & 0\\ \hline
2.6.1 & 6 & 2 & 0\\ \hline
2.6.2 & 2 & 1 & 5\\ \hline
2.7.1 & 5 & 1 & 2\\ \hline
2.7.2 & 3 & 3 & 2\\ \hline
2.7.3 & 3 & 5 & 0\\ \hline
2.8.1 & 6 & 2 & 0\\ \hline
2.9.1 & 2 & 6 & 0\\ \hline
2.9.2 & 1 & 6 & 1\\ \hline
2.9.3 & 1 & 7 & 0\\ \hline
3.1 & 3 & 5 & 0\\ \hline
3.2 & 2 & 2 & 4\\
\hline
\end{tabular}
\end{center}
\caption{Table showing the totals of positive, neutral and negative responses to each question.   Full results are available in \emph{Appendix C}.}
\label{tab:FeedbackScores}
\end{table}
\end{center}

This data can also be visualised using a diagram, as follows.

\begin{center}
\begin{figure}[H]
\begin{center}
\includegraphics[width=\textwidth]{figs/graphs/stacked-scores.png}
\end{center}
\caption{Diagram showing the proportions of positive, neutral and negative feedback per question.}
\label{fig:StackedScores}
\end{figure}
\end{center}

The modal feedback scores both by question and by topic are shown below, to determine which was the most common tone of response for each section.

\begin{center}
\begin{table}[H]
\begin{center}
\begin{tabular}{|c|r||r|}
\hline
\multicolumn{1}{|c|}{Topic} & \multicolumn{1}{c||}{Question} & \multicolumn{1}{c|}{Mode Score(s)}\\ \hline \hline
\multirow{3}{*}{Visibility \& Juxtaposability, Mode = 1} & 2.1.1 & 1\\
 & 2.1.2 & -1\\
 & 2.1.3 & 1\\ \hline
\multirow{3}{*}{Viscosity, Mode = 1} & 2.2.1 & 1\\
 & 2.2.2 & 1 and -1\\ \hline
\multirow{3}{*}{Error Proneness, Mode = -1} & 2.3.1 & -1\\
 & 2.3.2 & 0 and -1\\ \hline
\multirow{3}{*}{Closeness of Mapping, Mode = 1} & 2.4.1 & 1\\
 & 2.4.2 & -1\\ \hline
\multirow{3}{*}{Role Expressiveness, Mode = 1} & 2.5.1 & 1\\
 & 2.5.2 & 1\\
 & 2.5.3 & 1\\ \hline
\multirow{3}{*}{Hidden Dependencies, Mode = 1} & 2.6.1 & 1\\
 & 2.6.2 & -1\\ \hline
\multirow{3}{*}{Progressive Evaluation, Mode = 1} & 2.7.1 & 1\\
 & 2.7.2 & 1 and 0\\
 & 2.7.3 & 0\\ \hline
\multirow{1}{*}{Provisionality, Mode = 1} & 2.8.1 & 1\\ \hline
\multirow{3}{*}{Secondary Notation, Mode = 0} & 2.9.1 & 0\\
 & 2.9.2 & 0\\
 & 2.9.3 & 0\\ \hline
\multirow{3}{*}{General Feedback, Mode = 0} & 3.1 & 0\\
 & 3.2 & -1\\
\hline
\end{tabular}
\end{center}
\caption{Table showing the modal feedback values by question and by groups of questions.}
\end{table}
\end{center}

The modal feedback score across all questions was also taken, giving a value of 1, hence the feedback was broadly positive.\footnote{The total number of responses were: 39 negative, 50 neutral, 87 positive.}

By considering the modal values by topic, it is possible to determine roughly how well-received the software was, with regard to each aspect of the questionnaire.  Modes also varied by question, the consequences of which are noted within each section.

\newpage
\paragraph{Visibility \& Juxtaposability}\hfill

Mode = 1

Feedback in this section was broadly positive.  One participant described the notation as ``very intuitive to use'', although another noted that ``circles +[sic] operators could be larger''.  Questions 2.1.1 and 2.1.3 received mostly positive feedback; however in Question 2.2.2, several users noted that the connecting lines between circles could occasionally become confusing when numerous.

\paragraph{Viscosity}\hfill

Mode = 1

Feedback on the perceived level of viscosity in the software was also positive. Question 2.2.1 received mostly positive feedback, whereas Question 2.2.2 was bimodal, having equal numbers of positive and negative responses.  Users noted that it was ``very easy'' to make changes to previous work, and that it was ``easy to do and obvious how to do so''.  However, some felt that it was ``more difficult to get rid of arguments [using Ctrl]''.  Another user concurred with this: ``deleting additional numbers [...] is the most difficult, but still simple to remember''.

\paragraph{Error Proneness}\hfill

Mode = -1

Users responded to the questions in this section with negative feedback.  Several users felt that the drag-and-drop mechanism for joining circles was easily misunderstood, having tried to terminate lines in invalid locations, although one user notes that ``I didn't make any mistakes!''.

\paragraph{Closeness of Mapping}\hfill

Mode = 1

This section was generally met with positive feedback, but Question 2.4.2 received more negative reponses.  Participants felt that the notation was ``very close[ly]'' related to the result that they were describing, with one user going so far as to say that it was ``exactly so''.

Several users found that the limitation of available numbers to the range -10 to 10 to be a strange way of working.  This limitation was a conscious design choice, and it was expected that users would appreciate the need to prevent the software from becoming used as simply a calculator.  Additionally, it was difficult to explain this without imposing a method of calculation, and as such this limitation was not mentioned in the tutorial video.  Some participants therefore concluded that it was a flaw in the software.

\paragraph{Role Expressiveness}\hfill

Mode = 1

Feedback on this section was broadly positive, with Question 2.5.3 receiving particularly favourable responses.  User opinions differed with regard to the clarity of the notation.  A majority felt that it was ``well laid out'', and several users found the colour-coding ``useful''.  However, some found that they were liable to forget the meaning of some parts of the notation.  For example:
\begin{center}
\parbox[c]{\textwidth-2cm}{
\small
``if I were looking at a completed diagram it might be confusing but when you work through a question and break it down it is easy to see what's what because you create each part of it as you get there.''
}
\end{center}

Most users felt that there were no parts of the notation that were especially difficult to interpret, although a few noted that the screen can become busy for more complex calculations.  One user agrees with the feedback from the `Visibility \& Juxtaposability' section in stating that connecting lines between circles can become ``difficult to take in altogether'' when numerous.  All users responded to Question 2.5.3 with ``no'', ``none'' or equivalent answers, indicating that they only used notation that they felt was necessary.

\paragraph{Hidden Dependencies}\hfill

Mode = 1

Feedback on this section was mixed, though modally positive.  Question 2.6.1 received a positive response: most users found that dependencies between units of the notation were made very clear, apart from one user who found that, while dependencies were clear, it was difficult to infer the results of their calculations whilst constructing them.  One user neglected to answer the question.  However, the feedback on Question 2.6.2 was negative.  Once again, several users found that the notation became harder to read, and dependencies harder to follow, for large numbers of circles.  This suggests that either circles should be made smaller, the canvas made larger, or an alternative to linking lines, which can become ``tangled'', should be sought.  Zoom functionality (an unimplemented feature of the prototypes) would also have solved this problem.

\paragraph{Progressive Evaluation}\hfill

Mode = 1

Feedback here was modally positive, though responses to individual question were more mixed.  Users found the running total in the information panel to be a useful means of tracking their progress, but noted that it was difficult to determine the value of partially-completed calculations where several groups of circles were kept separate until the end, due to the single answer-display.

\paragraph{Provisionality}\hfill

Mode = 1

Response to this question was positive.  Users spoke favourably of the running total displayed in the information panel, coupled with the prominent bin icon to faciliate easy deletion.  One user comments:
\begin{center}
\parbox[c]{\textwidth-2cm}{
\small
``I could create notations [i.e. circles] and throw them in the bin if they didn't help.''
}
\end{center}
This is in accordance with the intention of the project, which is to allow users to explore potential solutions and to determine why a given method may or may not work.

\paragraph{Secondary Notation}\hfill

Mode = 0

Feedback here was neutral.  A means of making notes or comments is not provided in the software, though this was neither well- nor poorly-received by the users.  Several users mentioned that, given the opportunity to annotate their diagrams in paper form, they would: ``keep a track of the different sums I had done so far''.  Users commented that they did not add any extra marks in the software to clarify or emphasise the information displayed onscreen.

\newpage
\paragraph{General Feedback}\hfill

Mode = 0

Feedback in this section was largely neutral.  Question 3.1 received neutral responses, whereas Question 3.2 received negative feedback.  Most users did not feel that they were using the software in a way which was `wrong' or `unusual', though one user did feel that having to create larger numbers from the limited range available constituted such behaviour.  Furthermore, several users suggested that an improvement to the software would be to remove the limitation (see `Closeness of Mapping').  Others felt that the circles should be larger, or that the canvas should be more capacious.

\section{Summary}
% did it succeed?
% TODO: time render, log system time in ms before/after render
% by which of course i mean update-answer
A criterion for the success of the project was to develop an application that was able to represent calculations using a novel visual metaphor, and to be able to evalute and display the results of this in under 300ms.  This has been achieved, with the function \verb¬update-answer¬ taking an average of approximately 7ms to execute.

As an extension, a user study was conducted to assess the suitability of the software to complement traditional methods of mathematical education in the target audience of Year 5 pupils.  The results of this study can be summarised thus:
\begin{itemize}
\item There is a statistically-significant difference between the time taken to solve questions by hand and by using the software
\begin{itemize}
\item For simple questions, \emph{Nico} increases the solution time
\item For complex questions involving large numbers and nested calculations, \emph{Nico} decreases the solution time
\end{itemize}
\item \emph{Nico} does not negatively impact the user's ability to answer question correctly
\item Feedback on the software was broadly positive, with users concluding that:
\begin{itemize}
\item The notation had few hidden dependencies
\item With many circles, the canvas could sometimes become overcrowded
\item The software encourages exploration, and it is easy to try out many potential solutions
\item The limitation of numbers to the range -10 to 10 was confusing, especially as it was not explained
\end{itemize}
\end{itemize}

\cleardoublepage
\chapter{Conclusions}
% did it achieve what was set out in the proposal?  300ms, etc.?
% does it work?
% user study
% what could be improved?
% overall, is it actually useful?

Overall this project has been very successful.  Not only have the success criteria for this project been fulfilled, but a user study has also been conducted as an extension.  As quoted at the beginning of Chapter 3, the success criteria were to develop an application that is able to generate and evaluate expressions from a novel graphical notation, displaying results in less than 300ms.  In this regard, \emph{Nico} has met and far exceeded expectations.  A functioning application and accompanying visual metaphor have been developed that allow the user to express a calculation graphically.  From this, the application is able to generate a valid Clojure S-expression -- the tree -- that is able to be passed around and evaluated.  The application takes considerably less than 300ms to interpret the notation and update the display accordingly.

The software itself comprises a three-layer structure, with an infrastructure that interprets and manipulates user-input data; an interaction handler that provides an iterface between the infrastructure and the user interface; and the UI itself, the environment in which the user interacts with the notation to express calculations.

The user study conducted as an extension to the main project concluded that \emph{Nico} made a statistically-significant difference to the users' speed of answering questions.  This was particularly the case for more complex questions, with many subcalculations and larger numbers, the software was shown to decrease the time taken for users to answer.

User feedback was modally positive, with users finding that the software made it easy to explore many possible solutions, but also that the notation could become confusing where a large number of circles occupied much of the canvas' available area.  However, it was also noted that the notation had few \emph{hidden dependencies}, and was easy to read in most cases.

\section{Future Work}

The project suggests a number of potential improvements and extensions:
\begin{itemize}
\item To improve the notation in accordance with the feedback received
\item To reimplement deletion, preserving both references and referential integrity
\item To explore the effects of implementing a negative mark scheme, in which marks are deducted for each incorrect submission
\item To develop a partner application to allow tutors to design their own question sets
\item To consider other novel metaphors for calculation
\item To conduct a larger-scale user study with a similar premise, involving participants from the target audience
\item To conduct a study into the long-term effects of using such software in a real school environment
\end{itemize}

The application of Petre and Green's `Cognitive Dimensions' framework to metaphors in educational software might yield further worthwhile research, extending Blackwell and Green's research into visual programming languages into the domain of educational software \cite{Green1996} \cite{Blackwell1999}.

This project will be further developed and maintained in the future, with a view to its introduction into actual learning environments.




\cleardoublepage

%%%%%%%%%%%%%%%%%%%%%%%%%%%%%%%%%%%%%%%%%%%%%%%%%%%%%%%%%%%%%%%%%%%%%
% the bibliography

\addcontentsline{toc}{chapter}{Bibliography}
\bibliography{refs}
\cleardoublepage

%%%%%%%%%%%%%%%%%%%%%%%%%%%%%%%%%%%%%%%%%%%%%%%%%%%%%%%%%%%%%%%%%%%%%
% the appendices
\appendix

\chapter{Third-Party Tools}

A list of the third-party tools that were used in the development of the project follows.

\begin{itemize}
\item Ubuntu Linux 10.04, Arch Linux 2010.05, Microsoft Windows 7
\item Clojure 1.2.0
\item Leiningen 1.6.1.1
\item OpenJDK 6
\item Seesaw 1.3.1-SNAPSHOT
\item swank-clojure 1.3.4-SNAPSHOT
\item GNU Emacs 23.1.1
\item A modified version of Overtone's Emacs configuration \cite{OvertoneEmacsD}, including:--
\begin{itemize}
\item SLIME/SWANK (revision as of 15/10/2009)
\item clojure-mode 1.11.5
\item undo-tree 0.3.3
\end{itemize}
\item Git 1.7.0.4
\item GitHub
\item Balsamiq Mockups
\item Google Docs
\item R 2.15.0
\end{itemize}

\chapter{User Response Times}

The table and plots of the times taken for each user to answer each question follow.

\begin{center}
\begin{table}[H]
\begin{center}
\begin{tabular}{|r||r|r||r|r||r|r||r|r|}
\cline{2-9}
\multicolumn{1}{c|}{} & \multicolumn{8}{c|}{Group 1}\\ \cline{2-9}
\multicolumn{1}{c|}{} & \multicolumn{2}{c||}{Participant 1} & \multicolumn{2}{c||}{Participant 2} & \multicolumn{2}{c||}{Participant 3} & \multicolumn{2}{c|}{Participant 4}\\ \hline
\multicolumn{1}{|c||}{Question} & \multicolumn{1}{c|}{Hand} & \multicolumn{1}{c||}{\emph{Nico}} & \multicolumn{1}{c|}{Hand} & \multicolumn{1}{c||}{\emph{Nico}} & \multicolumn{1}{c|}{Hand} & \multicolumn{1}{c||}{\emph{Nico}} & \multicolumn{1}{c|}{Hand} & \multicolumn{1}{c|}{\emph{Nico}}\\ \hline \hline
Q1 & 4600 & 19547 & 3100 & 34094 & 1800 & 24657 & 1300 & 38891\\ \hline
Q2 & 8800 & 23313 & 7000 & 24329 & 6300 & 17437 & 5400 & 12031\\ \hline
Q3 & 20600 & 53500 & 13600 & 36360 & 10100 & 40798 & 10700 & 31813\\ \hline
Q4 & 36200 & 54375 & 26400 & 60156 & 14200 & 79704 & 23800 & 50094\\ \hline
Q5 & 61900 & 103344 & 39600 & 79844 & 26500 & 103221 & 33400 & 61064\\ \hline
Q6 & 213400 & 224406 & 190300 & 215376 & 84900 & 172940 & 181700 & 154691\\ \hline
Q7 & 235200 & 165813 & 1996000 & 103376 & 909000 & 48922 & 194200 & 52595\\ \hline
Q8 & 259500 & 306625 & 282500 & 158094 & 147200 & 138581 & 256700 & 155347\\ \hline
Q9 & 304400 & 230953 & 313500 & 123626 & 159800 & 183722 & 271700 & 196144\\ \hline
Q10 & 416500 & 307734 & 357700 & 213642 & 197100 & 245161 & 322200 & 235364\\ \hline
\multicolumn{9}{c}{}\\ \cline{2-9}
\multicolumn{1}{c|}{} & \multicolumn{8}{c|}{Group 2}\\ \cline{2-9}
\multicolumn{1}{c|}{} & \multicolumn{2}{c||}{Participant 1} & \multicolumn{2}{c||}{Participant 2} & \multicolumn{2}{c||}{Participant 3} & \multicolumn{2}{c|}{Participant 4}\\ \hline
\multicolumn{1}{|c||}{Question} & \multicolumn{1}{c|}{Hand} & \multicolumn{1}{c||}{\emph{Nico}} & \multicolumn{1}{c|}{Hand} & \multicolumn{1}{c||}{\emph{Nico}} & \multicolumn{1}{c|}{Hand} & \multicolumn{1}{c||}{\emph{Nico}} & \multicolumn{1}{c|}{Hand} & \multicolumn{1}{c|}{\emph{Nico}}\\ \hline \hline
Q1 & 1200 & 22299 & 2100 & 23516 & 2600 & 16142 & 2600 & 21985\\ \hline
Q2 & 2200 & 24530 & 5600 & 19015 & 7200 & 16875 & 7200 & 23422\\ \hline
Q3 & 17300 & 47620 & 7600 & 29735 & 18400 & 27923 & 15900 & 48470\\ \hline
Q4 & 23000 & 50704 & 20700 & 43094 & 25300 & 37516 & 38600 & 174160\\ \hline
Q5 & 46600 & 134828 & 33300 & 70313 & 35500 & 58954 & 72400 & 98846\\ \hline
Q6 & 199900 & 629835 & 136400 & 164891 & 99000 & 113659 & 266600 & 252708\\ \hline
Q7 & 202400 & 80277 & 142200 & 48890 & 1106000 & 52954 & 272900 & 72767\\ \hline
Q8 & 295900 & 248736 & 182000 & 107375 & 184700 & 78001 & 348900 & 257224\\ \hline
Q9 & 312900 & 152185 & 204100 & 118594 & 200500 & 67485 & 384500 & 144753\\ \hline
Q10 & 357300 & 238300 & 261600 & 203110 & 237500 & 324581 & 472900 & 270177\\
\hline
\end{tabular}
\end{center}
\caption{Table showing the time in milliseconds taken by each participant to complete each question, by hand and using the software.}
\label{tab:BigTimes}
\end{table}
\end{center}

\begin{center}
\begin{figure}[H]
\begin{center}
\includegraphics[height=\textheight/2-2cm]{figs/graphs/q1.png}
\end{center}
\caption{Time taken in milliseconds per participant to answer Question 1.}
\end{figure}
\end{center}

\begin{center}
\begin{figure}[H]
\begin{center}
\includegraphics[height=\textheight/2-2cm]{figs/graphs/q2.png}
\end{center}
\caption{Time taken in milliseconds per participant to answer Question 2.}
\end{figure}
\end{center}

\begin{center}
\begin{figure}[H]
\begin{center}
\includegraphics[height=\textheight/2-2cm]{figs/graphs/q3.png}
\end{center}
\caption{Time taken in milliseconds per participant to answer Question 3.}
\end{figure}
\end{center}

\begin{center}
\begin{figure}[H]
\begin{center}
\includegraphics[height=\textheight/2-2cm]{figs/graphs/q4.png}
\end{center}
\caption{Time taken in milliseconds per participant to answer Question 4.}
\end{figure}
\end{center}

\begin{center}
\begin{figure}[H]
\begin{center}
\includegraphics[height=\textheight/2-2cm]{figs/graphs/q5.png}
\end{center}
\caption{Time taken in milliseconds per participant to answer Question 5.}
\end{figure}
\end{center}

\begin{center}
\begin{figure}[H]
\begin{center}
\includegraphics[height=\textheight/2-2cm]{figs/graphs/q6.png}
\end{center}
\caption{Time taken in milliseconds per participant to answer Question 6.}
\end{figure}
\end{center}

\begin{center}
\begin{figure}[H]
\begin{center}
\includegraphics[height=\textheight/2-2cm]{figs/graphs/q7.png}
\end{center}
\caption{Time taken in milliseconds per participant to answer Question 7.}
\end{figure}
\end{center}

\begin{center}
\begin{figure}[H]
\begin{center}
\includegraphics[height=\textheight/2-2cm]{figs/graphs/q8.png}
\end{center}
\caption{Time taken in milliseconds per participant to answer Question 8.}
\end{figure}
\end{center}

\begin{center}
\begin{figure}[H]
\begin{center}
\includegraphics[height=\textheight/2-2cm]{figs/graphs/q9.png}
\end{center}
\caption{Time taken in milliseconds per participant to answer Question 9.}
\end{figure}
\end{center}

\begin{center}
\begin{figure}[H]
\begin{center}
\includegraphics[height=\textheight/2-2cm]{figs/graphs/q10.png}
\end{center}
\caption{Time taken in milliseconds per participant to answer Question 10.}
\end{figure}
\end{center}

\chapter{User Feedback}

The results of the user feedback are shown below, with -1 indicating a negative response, 0 indicating a neutral response and 1 indicating a positive response.

\begin{center}
\begin{table}[H]
\begin{center}
\begin{tabular}{|r||r|r|r|r||r|r|r|r||r|r|r|}
\cline{2-12}
\multicolumn{1}{c}{} & \multicolumn{4}{|c|}{Group 1} & \multicolumn{4}{c||}{Group 2} & \multicolumn{3}{c|}{Totals}\\ \hline
\multicolumn{1}{|c||}{Question} & \multicolumn{1}{c|}{P1} & \multicolumn{1}{c|}{P2} & \multicolumn{1}{c|}{P3} & \multicolumn{1}{c||}{P4} & \multicolumn{1}{c|}{P1} & \multicolumn{1}{c|}{P2} & \multicolumn{1}{c|}{P3} & \multicolumn{1}{c||}{P4} & \multicolumn{1}{c|}{+} & \multicolumn{1}{c|}{±} & \multicolumn{1}{c|}{-}\\ \hline \hline
2.1.1 & 1 & 1 & -1 & 0 & 1 & 1 & 1 & 1 & 6 & 1 & 1\\ \hline
2.1.2 & -1 & 1 & -1 & -1 & -1 & -1 & 1 & -1 & 2 & 0 & 6\\ \hline
2.1.3 & 1 & 1 & 1 & 1 & 1 & 1 & 1 & 1 & 8 & 0 & 0\\ \hline
2.2.1 & 1 & 1 & 1 & 0 & 1 & 1 & 1 & 1 & 7 & 1 & 0\\ \hline
2.2.2 & -1 & 1 & -1 & -1 & -1 & 1 & 1 & 1 & 4 & 0 & 4\\ \hline
2.3.1 & -1 & -1 & -1 & -1 & -1 & 1 & 0 & 0 & 1 & 2 & 5\\ \hline
2.3.2 & -1 & 1 & 0 & -1 & 0 & 0 & -1 & 1 & 2 & 3 & 3\\ \hline
2.4.1 & 1 & 1 & -1 & 1 & -1 & 1 & 1 & 1 & 6 & 0 & 2\\ \hline
2.4.2 & -1 & -1 & -1 & -1 & -1 & 1 & 1 & 1 & 3 & 0 & 5\\ \hline
2.5.1 & -1 & 1 & -1 & 1 & 1 & 1 & 1 & 1 & 6 & 0 & 2\\ \hline
2.5.2 & 0 & 1 & 0 & 0 & -1 & 1 & 1 & 1 & 4 & 3 & 1\\ \hline
2.5.3 & 0 & 1 & 1 & 1 & 0 & 1 & 1 & 1 & 6 & 2 & 0\\ \hline
2.6.1 & 1 & 1 & 0 & 0 & 1 & 1 & 1 & 1 & 6 & 2 & 0\\ \hline
2.6.2 & -1 & -1 & -1 & -1 & -1 & 1 & 0 & 1 & 2 & 1 & 5\\ \hline
2.7.1 & -1 & 1 & -1 & 1 & 0 & 1 & 1 & 1 & 5 & 1 & 2\\ \hline
2.7.2 & 0 & -1 & -1 & 1 & 1 & 0 & 1 & 0 & 3 & 3 & 2\\ \hline
2.7.3 & 0 & 1 & 0 & 1 & 0 & 0 & 1 & 0 & 3 & 5 & 0\\ \hline
2.8.1 & 1 & 1 & 1 & 0 & 1 & 1 & 1 & 0 & 6 & 2 & 0\\ \hline
2.9.1 & 1 & 0 & 0 & 0 & 0 & 1 & 0 & 0 & 2 & 6 & 0\\ \hline
2.9.2 & 0 & 0 & -1 & 0 & 0 & 1 & 0 & 0 & 1 & 6 & 1\\ \hline
2.9.3 & 0 & 0 & 0 & 1 & 0 & 0 & 0 & 0 & 1 & 7 & 0\\ \hline
3.1 & 0 & 0 & 1 & 0 & 0 & 0 & 1 & 1 & 3 & 5 & 0\\ \hline
3.2 & 0 & 1 & -1 & 0 & -1 & 1 & -1 & -1 & 2 & 2 & 4\\
\hline
\end{tabular}
\end{center}
\caption{Table showing the tone of feedback by participant and question, with the totals for each type of feedback by question.}
\end{table}
\end{center}

\chapter{Questionnaire}

The questionnaire used in the user study follows.

\includepdf[pages=-]{qs.pdf}

\cleardoublepage

\chapter{Project Proposal}

The original project proposal follows.

\includepdf[pages=-]{proposal_rev04.pdf}
% \input{propbody}

\end{document}
